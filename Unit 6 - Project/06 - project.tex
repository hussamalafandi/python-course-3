\documentclass[11pt,a4paper]{article}

% =========================
% Packages
% =========================
\usepackage[a4paper,margin=2.3cm,headheight=14pt]{geometry}
\usepackage[T1]{fontenc}
\usepackage[utf8]{inputenc} % remove if using lualatex/xelatex
\usepackage{lmodern}
\usepackage{microtype}
\usepackage{parskip}        % nicer paragraph spacing
\usepackage{enumitem}
\usepackage{booktabs}
\usepackage{tabularx}
\usepackage{xcolor}
\usepackage{hyperref}
\usepackage{fancyhdr}
\usepackage{titlesec}
\usepackage{listings}
\usepackage{graphicx}

% =========================
% Styling
% =========================
\definecolor{accent}{HTML}{0B3D91}
\definecolor{muted}{HTML}{6B7280}
\definecolor{backcolour}{HTML}{F6F7F9}
\definecolor{codegreen}{rgb}{0,0.55,0}
\definecolor{codegray}{rgb}{0.5,0.5,0.5}
\definecolor{codepurple}{rgb}{0.58,0,0.82}

\hypersetup{
  colorlinks=true,
  linkcolor=accent,
  urlcolor=accent,
  citecolor=accent
}

\titleformat{\section}{\large\bfseries\color{accent}}{}{0pt}{}
\titleformat{\subsection}{\normalsize\bfseries\color{accent}}{}{0pt}{}
\titleformat{\subsubsection}{\normalsize\bfseries}{}{0pt}{}

\setlist[itemize]{leftmargin=*, itemsep=0pt, topsep=2pt, parsep=0pt}
\setlist[enumerate]{leftmargin=*, itemsep=0pt, topsep=2pt, parsep=0pt}

\lstdefinestyle{mystyle}{
  backgroundcolor=\color{backcolour},
  commentstyle=\color{codegreen},
  keywordstyle=\color{accent},
  numberstyle=\tiny\color{codegray},
  stringstyle=\color{codepurple},
  basicstyle=\ttfamily\small,
  breaklines=true,
  numbers=left,
  numbersep=6pt,
  showstringspaces=false,
  tabsize=2,
  frame=single,
  rulecolor=\color{black!10}
}
\lstset{style=mystyle}

\pagestyle{fancy}
\fancyhf{}
\lhead{\textbf{Unit 6 --- Mini Project I}}
\rhead{\textcolor{muted}{Expense Tracker \& Reporting (CLI)}}
\cfoot{\thepage}

% =========================
% Document Meta
% =========================
\newcommand{\courseTitle}{Python Programming \& Data Analysis}
\newcommand{\unitTitle}{Unit 6 --- Mini Project I: Programming Foundations}
\newcommand{\projectTitle}{Expense Tracker \& Reporting (CLI)}
\newcommand{\instructorName}{Hussam Alafandi}
\newcommand{\handoverDate}{\today}

\begin{document}

% =========================
% Header block
% =========================
{\Large \textbf{\courseTitle}}\\[-2pt]
{\large \textbf{\unitTitle}}\\[4pt]
{\normalsize \textcolor{muted}{Project Handout --- \projectTitle}}\\[10pt]

\begin{tabularx}{\textwidth}{@{}lX@{}}
% \textbf{Instructor:} & \instructorName \\
\textbf{Date:} & \handoverDate \\
\textbf{Estimated workload:} & 6--10 hours (depending on optional extensions) \\
\textbf{Allowed libraries:} & Python \textbf{Standard Library only} (no external packages) \\
\end{tabularx}

\vspace{10pt}
\hrule
\vspace{12pt}

% =========================
% 1. Overview
% =========================
\section{Project Overview}

In this mini project, you will build a \textbf{command-line (CLI) expense tracker} that allows users to record income and expenses, view and filter entries, generate basic summary reports, and persist data to disk (\texttt{CSV} or \texttt{JSON}).

\textbf{Primary goal:} consolidate the foundational skills from Units 1--5:
\begin{itemize}
  \item Tooling + running scripts (Unit 1)
  \item Types, conversion, I/O, debugging (Unit 2)
  \item Control flow, loops, validation, menu programs (Unit 3)
  \item Data structures + aggregation without Pandas (Unit 4)
  \item Functions, modularity, imports, multi-file programs (Unit 5)
\end{itemize}

\textbf{Forward compatibility:} this project is intentionally designed to be refactored later into:
\begin{itemize}
  \item \textbf{OOP} (Units 7--8): \texttt{Transaction}, \texttt{Ledger}, \texttt{Report}
  \item \textbf{Algorithms} (Unit 9): sorting, searching
  \item \textbf{Pandas/Viz} (Units 11--12): load CSV, groupby, charts
\end{itemize}

% =========================
% 2. Learning objectives
% =========================
\section{Learning Objectives}

By completing this project, you will demonstrate that you can:
\begin{itemize}
  \item Design a program with a clear logical flow and robust user input handling
  \item Use lists, dictionaries, and sets to represent and aggregate structured data
  \item Decompose a program into reusable functions and modules
  \item Read/write structured data using the standard library (\texttt{csv} or \texttt{json})
  \item Produce readable and maintainable code (naming, structure, separation of concerns)
\end{itemize}

% =========================
% 3. Functional requirements
% =========================
\newpage
\section{Functional Requirements (MVP)}

Your program must implement the following features:

\subsection{1) Add transaction}
The user can add a transaction with:
\begin{itemize}
  \item \textbf{date} (string, recommended format: \texttt{YYYY-MM-DD})
  \item \textbf{type} (\texttt{income} or \texttt{expense})
  \item \textbf{category} (e.g., \texttt{food}, \texttt{rent}, \texttt{salary})
  \item \textbf{amount} (positive float)
  \item \textbf{note} (optional short text)
\end{itemize}
You must validate user input (no crashes on invalid input).

\subsection{2) List transactions}
The user can list all transactions. Minimum output: one line per transaction.

\textbf{Optional filters (recommended):} by category and/or type.

\subsection{3) Summary report}
Provide at least:
\begin{itemize}
  \item total income
  \item total expenses
  \item balance (= income - expenses)
  \item expense totals per category
\end{itemize}

\subsection{4) Persist data}
Your program must support \textbf{save and load} using either:
\begin{itemize}
  \item \textbf{CSV} (recommended; best for later Pandas units), or
  \item \textbf{JSON}
\end{itemize}

\textbf{Requirement:} when starting, attempt to load existing data; when quitting, save the current ledger.

% =========================
% 4. Data model
% =========================
\enlargethispage{2\baselineskip}
\section{Data Model (No external libraries)}

Represent a transaction as a dictionary and store transactions in a list:

\begin{lstlisting}[language=Python]
transaction = {
  "date": "2026-01-15",
  "type": "expense",     # "income" or "expense"
  "category": "food",
  "amount": 12.50,
  "note": "lunch"
}

ledger = [transaction1, transaction2, ...]
\end{lstlisting}
\nopagebreak[4]

\textbf{Notes:}
\nopagebreak[4]
\begin{itemize}
  \item Keep \texttt{amount} as a numeric type (\texttt{float}) in memory.
  \item Use a \texttt{set} where useful (e.g., extracting unique categories).
\end{itemize}

% =========================
% 5. Recommended structure
% =========================
\section{Code Organization (Required)}

This unit evaluates structure and modularity. Use a multi-file layout.

\subsection{Recommended folder structure}
\begin{lstlisting}[language=bash]
expense_tracker/
  main.py
  utils.py
  logic.py
  reports.py
  storage.py
  data/
    ledger.csv
\end{lstlisting}

\subsection{Module responsibilities (guideline)}
\begin{tabularx}{\textwidth}{@{}lX@{}}
\toprule
\textbf{File} & \textbf{Responsibility} \\
\midrule
\texttt{main.py} & Menu loop, orchestration, user interaction \\
\texttt{utils.py} & Input helpers (validation), formatting helpers \\
\texttt{logic.py} & Add/filter/update operations on the ledger \\
\texttt{reports.py} & Aggregations: totals, per-category summaries \\
\texttt{storage.py} & Load/save ledger via CSV or JSON \\
\bottomrule
\end{tabularx}

\subsection{Separation of concerns (important)}
\begin{itemize}
  \item Keep business logic \textbf{separate} from printing/input where possible.
  \item Prefer functions that \textbf{return values} instead of printing inside them.
\end{itemize}

% =========================
% 6. Persistence format
% =========================
\newpage
\section{Persistence Format}

\subsection{CSV (recommended)}
Use columns:
\texttt{date,type,category,amount,note}

Example:
\begin{lstlisting}[language=bash]
date,type,category,amount,note
2026-01-15,expense,food,12.50,lunch
2026-01-15,income,salary,2500.00,January
\end{lstlisting}

\subsection{Implementation sketch --- optional}
\begin{lstlisting}[language=Python]
import csv

FIELDS = ["date", "type", "category", "amount", "note"]

def save_csv(path, ledger):
    with open(path, "w", newline="", encoding="utf-8") as f:
        w = csv.DictWriter(f, fieldnames=FIELDS)
        w.writeheader()
        for tx in ledger:
            w.writerow(tx)

def load_csv(path):
    ledger = []
    try:
        with open(path, newline="", encoding="utf-8") as f:
            r = csv.DictReader(f)
            for row in r:
                row["amount"] = float(row["amount"])
                ledger.append(row)
    except FileNotFoundError:
        pass
    return ledger
\end{lstlisting}

% =========================
% 7. Milestones
% =========================
\section{Milestones}

\begin{enumerate}
  \item \textbf{Milestone 1 (MVP --- in memory)}: menu + add + list + validation
  \item \textbf{Milestone 2 (Reporting)}: totals + balance + per-category expenses
  \item \textbf{Milestone 3 (Persistence)}: load on start, save on exit
  \item \textbf{Milestone 4 (Optional extensions)}:
    \begin{itemize}
      \item sort by date or amount
      \item filter by type/category
      \item search keyword in \texttt{note}
      \item monthly report (group by \texttt{YYYY-MM})
      \item export \texttt{report.txt}
    \end{itemize}
\end{enumerate}

% =========================
% 8. Quality checklist
% =========================
\section{Quality Checklist (What we look for)}

Before submission, verify:
\begin{itemize}
  \item Program does not crash on invalid input (robust validation)
  \item Data loads/saves correctly (no loss, correct amount conversion)
  \item Code is structured into functions and modules (no giant \texttt{main})
  \item Names are meaningful; repetition is minimized
  \item Output is readable (consistent formatting)
\end{itemize}

% =========================
% 9. Submission
% =========================
\section{Submission Instructions}

Submit a single folder (zip) containing:
\begin{itemize}
  \item All \texttt{.py} files
  \item \texttt{data/ledger.csv} or \texttt{data/ledger.json} (with sample entries)
  \item A short \texttt{README.md} including:
    \begin{itemize}
      \item setup instructions (venv/conda)
      \item how to run the program
      \item implemented features + optional extensions
    \end{itemize}
\end{itemize}

\textbf{Run command (example):}
\begin{lstlisting}[language=bash]
python main.py
\end{lstlisting}

% =========================
% 11. Appendix: Suggested menu
% =========================
\section{Appendix: Suggested Menu (CLI)}

You may implement any reasonable menu, but the following works well:

\begin{lstlisting}[language={}]
1) Add transaction
2) List transactions
3) Summary report
4) Save and quit
\end{lstlisting}

\textbf{Guideline:} invalid choices should show a short message and return to the menu.

% =========================
% Footer note
% =========================
\vspace{10pt}
\hrule
\vspace{8pt}
\textcolor{muted}{\small Tip: Keep your core logic in functions that return values.}

\end{document}
