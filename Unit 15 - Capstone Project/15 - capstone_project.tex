\documentclass[11pt,a4paper]{article}

% =========================
% Packages
% =========================
\usepackage[a4paper,margin=2.3cm,headheight=14pt]{geometry}
\usepackage[T1]{fontenc}
\usepackage[utf8]{inputenc} % remove if using lualatex/xelatex
\usepackage{lmodern}
\usepackage{microtype}
\usepackage{parskip}        % nicer paragraph spacing
\usepackage{enumitem}
\usepackage{booktabs}
\usepackage{tabularx}
\usepackage{xcolor}
\usepackage{hyperref}
\usepackage{fancyhdr}
\usepackage{titlesec}
\usepackage{listings}
\usepackage{graphicx}
\usepackage{multirow}
\usepackage{amsmath}

% =========================
% Styling
% =========================
\definecolor{accent}{HTML}{0B3D91}
\definecolor{muted}{HTML}{6B7280}
\definecolor{backcolour}{HTML}{F6F7F9}
\definecolor{codegreen}{rgb}{0,0.55,0}
\definecolor{codegray}{rgb}{0.5,0.5,0.5}
\definecolor{codepurple}{rgb}{0.58,0,0.82}

\hypersetup{
  colorlinks=true,
  linkcolor=accent,
  urlcolor=accent,
  citecolor=accent
}

\titleformat{\section}{\large\bfseries\color{accent}}{}{0pt}{}
\titleformat{\subsection}{\normalsize\bfseries\color{accent}}{}{0pt}{}
\titleformat{\subsubsection}{\normalsize\bfseries}{}{0pt}{}

\setlist[itemize]{leftmargin=*, itemsep=0pt, topsep=2pt, parsep=0pt}
\setlist[enumerate]{leftmargin=*, itemsep=0pt, topsep=2pt, parsep=0pt}

\lstdefinestyle{mystyle}{
  backgroundcolor=\color{backcolour},
  commentstyle=\color{codegreen},
  keywordstyle=\color{accent},
  numberstyle=\tiny\color{codegray},
  stringstyle=\color{codepurple},
  basicstyle=\ttfamily\small,
  breaklines=true,
  numbers=left,
  numbersep=6pt,
  showstringspaces=false,
  tabsize=2,
  frame=single,
  rulecolor=\color{black!10}
}
\lstset{style=mystyle}

\pagestyle{fancy}
\fancyhf{}
\lhead{\textbf{Capstone Project}}
\rhead{\textcolor{muted}{CityBike --- Bike-Sharing Analytics}}
\cfoot{\thepage}

% =========================
% Document Meta
% =========================
\newcommand{\courseTitle}{Python Programming \& Data Analysis}
\newcommand{\unitTitle}{Unit 15 --- Capstone Project}
\newcommand{\projectTitle}{CityBike --- Bike-Sharing Operations \& Usage Analytics Platform}
\newcommand{\instructorName}{Hussam Alafandi}
\newcommand{\handoverDate}{\today}

\begin{document}

% =========================
% Header block
% =========================
{\Large \textbf{\courseTitle}}\\[-2pt]
{\large \textbf{\unitTitle}}\\[4pt]
{\normalsize \textcolor{muted}{Project Handout --- \projectTitle}}\\[10pt]

\begin{tabularx}{\textwidth}{@{}lX@{}}
\textbf{Date:}               & \handoverDate \\
\textbf{Estimated workload:} & 20--30 hours \\
\textbf{Required libraries:} & \texttt{pandas}, \texttt{numpy}, \texttt{matplotlib} \\
\end{tabularx}

\vspace{10pt}
\hrule
\vspace{12pt}

% =========================
% 1. Motivation
% =========================
\section{Motivation \& Context}

Bike-sharing systems have become a cornerstone of modern urban transport in cities worldwide --- from Berlin's \emph{Nextbike} to New York's \emph{Citi Bike} and London's \emph{Santander Cycles}. These systems generate vast amounts of operational data every day: trip records, station usage, fleet maintenance logs, and user activity.

Managing such a system is a real-world software engineering challenge. Operators need to:
\begin{itemize}
  \item Track hundreds of bikes across dozens of stations
  \item Understand when and where demand peaks occur
  \item Schedule preventive maintenance to minimize downtime
  \item Analyze user behavior to optimize station placement and pricing
  \item Generate reports for city authorities and stakeholders
\end{itemize}

In this capstone project, you will build a \textbf{complete backend platform} for a fictional city bike-sharing service. You will model the domain with object-oriented design, load and clean operational datasets, implement algorithms, compute statistics, generate visualizations, and deliver a well-structured, version-controlled codebase.

% =========================
% 2. Overview
% =========================
\section{Project Overview}

This is the \textbf{final integrative project} of the course. It is designed to demonstrate mastery of every major topic covered across Units 1--13.

\textbf{This project integrates:}
\begin{itemize}
  \item Modular code structure, functions, multi-file organization (Units 1--5)
  \item Object-oriented design: classes, inheritance, abstract base classes, design patterns (Units 7--8)
  \item Algorithms: sorting, searching, complexity analysis (Unit 9)
  \item NumPy for numerical computing (Unit 10)
  \item Pandas for data analysis, cleaning, and aggregation (Unit 11)
  \item Data visualization with Matplotlib (Unit 12)
  \item Version control with Git and GitHub
\end{itemize}

Unlike the earlier mini projects, this capstone requires you to demonstrate \textbf{all} of the above in a single, cohesive application. Quality expectations are higher: your code must be documented, modular, and professionally structured.

% =========================
% 3. Functional requirements
% =========================
\newpage
\section{Functional Requirements}

\subsection{1) Object-Oriented Data Models \textnormal{\textcolor{muted}{(Units 7--8)}}}

Design a class hierarchy to represent the bike-sharing domain.

\subsubsection{Required Classes}
\begin{itemize}
  \item \textbf{\texttt{Entity}} (abstract base class): shared \texttt{id}, \texttt{created\_at}; enforces \texttt{\_\_str\_\_} and \texttt{\_\_repr\_\_} in subclasses
  \item \textbf{\texttt{Bike(Entity)}}: bike\_id, bike\_type, status (\texttt{available}/\texttt{in\_use}/\texttt{maintenance})
    \begin{itemize}
      \item \textbf{\texttt{ClassicBike(Bike)}}: gear\_count
      \item \textbf{\texttt{ElectricBike(Bike)}}: battery\_level, max\_range\_km
    \end{itemize}
  \item \textbf{\texttt{Station(Entity)}}: station\_id, name, capacity, latitude, longitude
  \item \textbf{\texttt{User(Entity)}}: user\_id, name, email, user\_type
    \begin{itemize}
      \item \textbf{\texttt{CasualUser(User)}}: day\_pass\_count
      \item \textbf{\texttt{MemberUser(User)}}: membership\_start, membership\_end, tier (\texttt{basic}/\texttt{premium})
    \end{itemize}
  \item \textbf{\texttt{Trip}}: trip\_id, user, bike, start\_station, end\_station, start\_time, end\_time, distance\_km
  \item \textbf{\texttt{MaintenanceRecord}}: record\_id, bike, date, maintenance\_type, cost, description
  \item \textbf{\texttt{BikeShareSystem}}: orchestrates loading, cleaning, analysis, and reporting
\end{itemize}

\subsubsection{OOP Requirements}
\begin{itemize}
  \item Proper \texttt{\_\_init\_\_} constructors with \textbf{input validation} (e.g., reject negative prices, invalid emails)
  \item String representations: \texttt{\_\_str\_\_} for user-friendly output, \texttt{\_\_repr\_\_} for debugging
  \item At least one \textbf{abstract base class} using Python's \texttt{abc} module
  \item At least one \textbf{inheritance hierarchy} with a minimum depth of two (e.g., \texttt{Entity} $\rightarrow$ \texttt{Bike} $\rightarrow$ \texttt{ElectricBike})
  \item Use of \textbf{properties} (\texttt{@property}) for controlled attribute access where appropriate
  \item Implement at least \textbf{two design patterns}:
    \begin{itemize}
      \item \textbf{Factory Pattern}: create \texttt{Bike} or \texttt{User} objects from CSV row dictionaries without exposing subclass constructors
      \item \textbf{Strategy Pattern}: interchangeable pricing strategies (e.g., \texttt{CasualPricing}, \texttt{MemberPricing}, \texttt{PeakHourPricing}) that compute trip cost
    \end{itemize}
\end{itemize}

\subsection{2) Data Loading \& Cleaning \textnormal{\textcolor{muted}{(Unit 11)}}}
\begin{itemize}
  \item Load the provided CSV files (\texttt{trips.csv}, \texttt{stations.csv}, \texttt{maintenance.csv}) into Pandas DataFrames
  \item Inspect structure, types, and missing values using \texttt{info()}, \texttt{describe()}, \texttt{isnull().sum()}
  \item Handle missing values with a \textbf{documented strategy} (drop, fill, interpolate --- justify your choice)
  \item Validate and convert types: parse dates, ensure numeric columns are numeric, standardize categorical values
  \item Remove duplicates and invalid entries (e.g., trips where end time is before start time)
  \item Export cleaned, analysis-ready datasets to CSV
\end{itemize}

\subsection{3) Algorithmic Analysis \textnormal{\textcolor{muted}{(Unit 9)}}}

Implement \textbf{both} of the following from scratch:

\begin{itemize}
  \item \textbf{Sorting}:
    \begin{itemize}
      \item Implement your own sorting algorithm (e.g., merge sort, quicksort)
      \item Apply it to sort trips by duration or distance
      \item Also sort using Python's built-in \texttt{sorted()} / Pandas \texttt{sort\_values()}
      \item Compare execution times using \texttt{timeit}
    \end{itemize}
  \item \textbf{Searching}:
    \begin{itemize}
      \item Implement your own search algorithm (e.g., binary search on sorted data)
      \item Apply it to find trips, stations, or users by ID or attribute
      \item Also search using built-in methods (e.g., \texttt{in}, Pandas \texttt{.loc[]}, \texttt{.query()})
      \item Compare execution times using \texttt{timeit}
    \end{itemize}
\end{itemize}

\textbf{Optional Analysis:}
\begin{itemize}
  \item Document the \textbf{time complexity (Big-O)} for each of your implementations
  \item Include a brief written comparison discussing the performance differences
  \item Explain why built-in functions are typically faster
\end{itemize}

\subsection{4) Numerical Computing \textnormal{\textcolor{muted}{(Unit 10)}}}

Use NumPy for at least the following:
\begin{itemize}
  \item Compute \textbf{distances} between stations using the Euclidean distance formula on latitude/longitude arrays:
    \[
      d = \sqrt{(\text{lat}_2 - \text{lat}_1)^2 + (\text{lon}_2 - \text{lon}_1)^2}
    \]
    (Use a simplified flat-earth model; no need for the Haversine formula.)
  \item Compute vectorized \textbf{statistics} on trip durations and distances (mean, median, standard deviation, percentiles) without using Python loops
  \item Use NumPy arrays for any batch numerical computation (e.g., fare calculation across all trips, z-score outlier detection)
\end{itemize}

\subsection{5) Analytics \& Business Insights \textnormal{\textcolor{muted}{(Units 10--11)}}}

Using Pandas and NumPy, answer at least \textbf{10} of the following questions:

\begin{enumerate}
  \item Total number of trips, total distance traveled, and average trip duration
  \item What are the \textbf{top 10 most popular} start stations and end stations?
  \item What are the \textbf{peak usage hours} during the day?
  \item Which \textbf{day of the week} has the highest trip volume?
  \item What is the \textbf{average trip distance} by user type (casual vs.\ member)?
  \item What is the \textbf{bike utilization rate} (percentage of time bikes are in use vs.\ available)?
  \item Show the \textbf{monthly trip trend} over time --- is ridership growing?
  \item Who are the \textbf{top 15 most active users} by trip count?
  \item What is the \textbf{total maintenance cost} per bike type (classic vs.\ electric)?
  \item What are the \textbf{most common station-to-station routes} (top 10 origin--destination pairs)?
  \item What is the \textbf{trip completion rate} (completed vs.\ cancelled trips)?
  \item What is the \textbf{average number of trips per user}, segmented by user type?
  \item Which bikes have the \textbf{highest maintenance frequency}?
  \item Identify \textbf{outlier trips} (unusually long/short duration or distance) using statistical methods
\end{enumerate}

\subsection{6) Visualization \textnormal{\textcolor{muted}{(Unit 12)}}}

Create at least \textbf{4 visualizations} using Matplotlib:
\begin{itemize}
  \item \textbf{Bar chart}: trips per station or revenue by user type
  \item \textbf{Line chart}: monthly trip volume trend over time
  \item \textbf{Histogram}: trip duration or distance distribution
  \item \textbf{Box plot}: trip duration comparison between user types or bike types
\end{itemize}

All charts must have:
\begin{itemize}
  \item Descriptive title
  \item Labeled axes with units
  \item Legend where applicable
  \item Clean, readable styling
\end{itemize}

Export all visualizations as \textbf{PNG files} to the \texttt{output/figures/} directory.

\subsection{7) Reporting \& Export \textnormal{\textcolor{muted}{(Units 5, 11--12)}}}
\begin{itemize}
  \item Export all cleaned datasets to CSV (\texttt{data/trips\_clean.csv}, etc.)
  \item Generate a \textbf{summary report} (\texttt{.txt}) containing key metrics and answers to business questions
  \item Export top stations, top users, and maintenance summaries as separate CSV or text files
\end{itemize}

% =========================
% 4. Clean Code
% =========================
\section{Clean Code \& Documentation}

\begin{itemize}
  \item \textbf{Docstrings}: every module, class, and public function must have a docstring
  \item \textbf{Naming}: use clear, descriptive names (no single-letter variables except in loops)
  \item \textbf{Separation of concerns}: business logic must be separate from I/O and display
  \item \textbf{DRY principle}: avoid code duplication; extract common logic into helper functions
  \item \textbf{Functions}: prefer functions that \textbf{return values} rather than printing inside them
  \item \textbf{Type hints}: add type annotations to all function signatures
  \item \textbf{No magic numbers}: use named constants for fixed values
\end{itemize}

% =========================
% Optional: Testing
% =========================
\subsection{Optional Extension: Unit Testing with \texttt{pytest} \textnormal{\textcolor{muted}{(Optional)}}}

This is \textbf{not required} but helps demonstrate best practices in software development. You can write unit tests for your OOP models and algorithms to ensure correctness and robustness.

If you choose to include tests, create a \texttt{tests/} directory and use \texttt{pytest}:

\begin{lstlisting}[language=bash]
citybike/
  tests/
    __init__.py
    test_models.py        # Tests for OOP models
    test_algorithms.py    # Tests for sorting & searching
\end{lstlisting}

\textbf{Guidelines:}
\begin{itemize}
  \item Write at least \textbf{10 test functions} covering model validation and algorithm correctness
  \item Use descriptive test names (e.g., \texttt{test\_electric\_bike\_rejects\_negative\_battery})
  \item Test both valid inputs (happy path) and invalid inputs (edge cases)
  \item All tests must pass when running:
\end{itemize}

\begin{lstlisting}[language=bash]
$ pytest tests/ -v
\end{lstlisting}

Add \texttt{pytest} to your \texttt{requirements.txt} if you include tests.

% =========================
% 6. Data model
% =========================
\section{Data Model}

You will work with \textbf{three CSV files}:

\subsection{trips.csv}
\begin{lstlisting}[language=bash]
trip_id,user_id,user_type,bike_id,bike_type,start_station_id,
end_station_id,start_time,end_time,duration_minutes,
distance_km,status
\end{lstlisting}

\subsection{stations.csv}
\begin{lstlisting}[language=bash]
station_id,station_name,capacity,latitude,longitude
\end{lstlisting}

\subsection{maintenance.csv}
\begin{lstlisting}[language=bash]
record_id,bike_id,bike_type,date,maintenance_type,cost,
description
\end{lstlisting}

\textbf{Notes:}
\begin{itemize}
  \item Data may contain missing values, duplicates, or inconsistencies (by design)
  \item \texttt{start\_time} and \texttt{end\_time} may need parsing to \texttt{datetime}
  \item \texttt{duration\_minutes} and \texttt{distance\_km} may be stored as strings; convert to float
  \item \texttt{status}: \texttt{completed} or \texttt{cancelled}
  \item \texttt{maintenance\_type}: \texttt{tire\_repair}, \texttt{brake\_adjustment}, \texttt{battery\_replacement}, \texttt{chain\_lubrication}, \texttt{general\_inspection}
  \item Some trips may have \texttt{end\_time} before \texttt{start\_time} --- these are invalid
\end{itemize}

% =========================
% 7. Code organization
% =========================
\section{Code Organization (Required)}

Use a multi-module project structure:

\begin{lstlisting}[language=bash]
citybike/
  main.py                 # Entry point: orchestration
  models.py               # OOP classes (Entity, Bike, Station, ...)
  analyzer.py             # BikeShareSystem: analysis methods
  algorithms.py           # Sorting, searching implementations
  numerical.py            # NumPy-based computations
  visualization.py        # Matplotlib chart functions
  pricing.py              # Strategy Pattern: pricing strategies
  factories.py            # Factory Pattern: object creation
  utils.py                # Validation, formatting, helpers
  data/
    trips.csv             # Raw trip data
    stations.csv          # Station metadata
    maintenance.csv       # Maintenance records
    trips_clean.csv       # Cleaned trip data (generated)
    stations_clean.csv    # Cleaned station data (generated)
  output/
    summary_report.txt    # Key metrics and insights
    top_stations.csv      # Most popular stations
    top_users.csv         # Most active users
    figures/              # Visualization PNGs
\end{lstlisting}

\textbf{Module responsibilities:}

\begin{tabularx}{\textwidth}{@{}lX@{}}
\toprule
\textbf{File} & \textbf{Responsibility} \\
\midrule
\texttt{main.py}          & Entry point; orchestrate the full pipeline from loading to reporting \\
\texttt{models.py}        & Domain classes with validation, inheritance, abstract base class \\
\texttt{analyzer.py}      & Data analysis: groupby, filtering, aggregation, business metrics \\
\texttt{algorithms.py}    & Custom sorting and searching implementations with benchmarks \\
\texttt{numerical.py}     & NumPy operations: distance calculations, statistics, outlier detection \\
\texttt{visualization.py} & All Matplotlib chart creation and export functions \\
\texttt{pricing.py}       & Strategy Pattern: pricing strategy interface and implementations \\
\texttt{factories.py}     & Factory Pattern: create model objects from raw data dictionaries \\
\texttt{utils.py}         & Input validation, date parsing, formatting helpers \\
\bottomrule
\end{tabularx}

\vspace{6pt}
\textbf{Design principles:}
\begin{itemize}
  \item Keep business logic separate from I/O and printing
  \item Prefer functions that return values instead of printing inside
  \item Each module should have a single, clear responsibility
  \item Write docstrings for all public functions and classes
  \item Use type hints in function signatures
\end{itemize}

% =========================
% 8. Git & GitHub
% =========================
\section{Version Control Requirements (Git \& GitHub)}

You \textbf{must} use Git for version control and host your project on GitHub.

\subsection{Repository Setup}
\begin{itemize}
  \item Create a new GitHub repository for this project
  \item Initialize with a \texttt{README.md} and \texttt{.gitignore} (Python template)
  \item Clone the repository locally and work from there
\end{itemize}

\subsection{Commit Requirements}
\begin{itemize}
  \item Make \textbf{at least 15 meaningful commits} throughout development
  \item Each commit should represent a logical unit of work
  \item Write clear, descriptive commit messages following conventional style:
    \begin{itemize}
      \item \texttt{feat: add Station class with capacity validation}
      \item \texttt{fix: handle missing values in trip duration column}
      \item \texttt{test: add unit tests for sorting algorithms}
      \item \texttt{docs: update README with setup instructions}
    \end{itemize}
\end{itemize}

\subsection{Branching (Required)}
\begin{itemize}
  \item Use feature branches for major components (e.g., \texttt{feature/models}, \texttt{feature/analysis}, \texttt{feature/visualization})
  \item Merge completed features into \texttt{main} via pull requests or local merge
  \item The \texttt{main} branch must always contain working code
\end{itemize}

\subsection{Required Files in Repository}
\begin{itemize}
  \item \texttt{README.md} --- project description, setup instructions, usage guide, and a brief summary of findings
  \item \texttt{.gitignore} --- exclude \texttt{\_\_pycache\_\_}, \texttt{.venv}, IDE files, \texttt{output/figures/*.png}
  \item \texttt{requirements.txt} --- list all dependencies (\texttt{pandas}, \texttt{numpy}, \texttt{matplotlib})
\end{itemize}

% =========================
% 9. Milestones
% =========================
\section{Milestones}

Follow these milestones to structure your development process. Each milestone should result in at least one commit.

\begin{enumerate}
  \item \textbf{Milestone 1 --- Project Setup} \\
    Create GitHub repo, set up project structure, add \texttt{README.md}, \texttt{.gitignore}, \texttt{requirements.txt}. Generate or place raw data files.
  \item \textbf{Milestone 2 --- Domain Models} \\
    Implement all OOP classes (\texttt{Entity}, \texttt{Bike}, \texttt{Station}, \texttt{User}, \texttt{Trip}, \texttt{MaintenanceRecord}) with validation, inheritance, and string representations. Implement Factory and Strategy patterns.
  \item \textbf{Milestone 3 --- Data Loading \& Cleaning} \\
    Load CSV files into DataFrames. Inspect, clean, validate, and export cleaned datasets.
  \item \textbf{Milestone 4 --- Algorithms} \\
    Implement custom sorting and searching algorithms. Benchmark against built-in functions. Document Big-O analysis.
  \item \textbf{Milestone 5 --- Numerical Computing} \\
    Implement NumPy-based distance calculations, statistical summaries, and outlier detection.
  \item \textbf{Milestone 6 --- Analytics} \\
    Answer all 10+ business questions using Pandas and NumPy. Generate summary report.
  \item \textbf{Milestone 7 --- Visualization} \\
    Create 4+ charts with proper labels and styling. Export as PNG.
  \item \textbf{Milestone 8 --- Polish \& Delivery} \\
    Add docstrings and type hints throughout. Update \texttt{README.md} with final instructions and findings summary. Final review and commit.
\end{enumerate}

% =========================
% 10. Quality checklist
% =========================
\newpage
\section{Quality Checklist}

Before submission, verify every item:

\textbf{Code Quality \& Clean Code:}
\begin{itemize}
  \item Code is organized into multiple modules with clear responsibilities
  \item All public functions and classes have docstrings
  \item Function signatures include type hints
  \item No magic numbers; named constants are used
  \item No code duplication; helpers are extracted
  \item No crashes on invalid input
\end{itemize}

\textbf{OOP Design:}
\begin{itemize}
  \item Abstract base class using \texttt{abc} module
  \item Inheritance hierarchy with at least 2 levels
  \item Constructors validate input
  \item \texttt{\_\_str\_\_} and \texttt{\_\_repr\_\_} implemented on all domain classes
  \item Properties (\texttt{@property}) used where appropriate
  \item Factory Pattern implemented and used
  \item Strategy Pattern implemented and used
\end{itemize}

\textbf{Data Analysis:}
\begin{itemize}
  \item All three CSV files properly loaded, inspected, and cleaned
  \item Missing value strategy documented
  \item NumPy used for numerical computations (not Python loops)
  \item Pandas used for aggregation, filtering, and grouping
  \item At least 10 business questions answered with code and output
\end{itemize}

\textbf{Algorithms:}
\begin{itemize}
  \item Custom sorting algorithm implemented and correct
  \item Custom searching algorithm implemented and correct
  \item Built-in equivalents also used for comparison
  \item Performance comparison with \texttt{timeit} included
  \item Big-O complexity documented for each algorithm
\end{itemize}

\textbf{Visualization:}
\begin{itemize}
  \item At least 4 charts created
  \item All charts have title, axis labels, and legend (where applicable)
  \item Charts are readable and professionally styled
  \item Exported as PNG files
\end{itemize}

\textbf{Git \& GitHub:}
\begin{itemize}
  \item Repository is public (or shared with instructor)
  \item At least 15 meaningful commits with clear messages
  \item Feature branches used for development
  \item \texttt{README.md} with project description, setup, and usage
  \item \texttt{requirements.txt} included
  \item \texttt{.gitignore} properly configured
\end{itemize}

% =========================
% 11. Data generator
% =========================
\newpage
\section{Appendix: Synthetic Data Generator}

Use the following script to generate the three dataset files. You may also modify or extend the generator to create additional scenarios.

\begin{lstlisting}[language=Python]
import pandas as pd
import numpy as np
from datetime import datetime, timedelta

np.random.seed(42)

# --- Station data ---
station_names = [
    "Central Station", "University Campus", "City Hall",
    "Riverside Park", "Market Square", "Tech Hub",
    "Old Town", "Harbor View", "Sports Arena",
    "West End", "North Gate", "Museum Quarter",
    "Business District", "Lakeside", "Airport Terminal"
]

stations = []
for i, name in enumerate(station_names):
    stations.append({
        "station_id": f"ST{100 + i}",
        "station_name": name,
        "capacity": np.random.choice([10, 15, 20, 25, 30]),
        "latitude": round(48.75 + np.random.uniform(0, 0.15), 6),
        "longitude": round(9.15 + np.random.uniform(0, 0.15), 6),
    })

stations_df = pd.DataFrame(stations)
stations_df.to_csv("stations.csv", index=False)

# --- Trip data ---
n_trips = 1500
user_ids = [f"USR{np.random.randint(1000, 1200)}"
            for _ in range(80)]
bike_ids = [f"BK{np.random.randint(200, 350)}"
            for _ in range(60)]
start_date = datetime(2024, 1, 1)

trips = []
for i in range(n_trips):
    user_type = np.random.choice(
        ["casual", "member"], p=[0.35, 0.65]
    )
    bike_type = np.random.choice(
        ["classic", "electric"], p=[0.6, 0.4]
    )
    start_st = np.random.choice(stations_df["station_id"])
    end_st = np.random.choice(stations_df["station_id"])
    start_time = start_date + timedelta(
        days=np.random.randint(0, 365),
        hours=np.random.randint(6, 23),
        minutes=np.random.randint(0, 60),
    )
    duration = max(2, np.random.exponential(25))
    end_time = start_time + timedelta(minutes=duration)
    distance = round(np.random.uniform(0.5, 15.0), 2)
    status = np.random.choice(
        ["completed", "cancelled", np.nan],
        p=[0.82, 0.12, 0.06]
    )

    trips.append({
        "trip_id": f"TR{10000 + i}",
        "user_id": np.random.choice(user_ids),
        "user_type": user_type,
        "bike_id": np.random.choice(bike_ids),
        "bike_type": bike_type,
        "start_station_id": start_st,
        "end_station_id": end_st,
        "start_time": start_time.strftime(
            "%Y-%m-%d %H:%M:%S"
        ),
        "end_time": end_time.strftime(
            "%Y-%m-%d %H:%M:%S"
        ),
        "duration_minutes": round(duration, 1),
        "distance_km": distance,
        "status": status,
    })

trips_df = pd.DataFrame(trips)

# Inject some messiness
idx = np.random.choice(n_trips, 30, replace=False)
trips_df.loc[idx[:10], "duration_minutes"] = np.nan
trips_df.loc[idx[10:20], "distance_km"] = np.nan
trips_df.loc[idx[20:25], "end_time"] = (
    trips_df.loc[idx[20:25], "start_time"]
)
dup_rows = trips_df.sample(15)
trips_df = pd.concat(
    [trips_df, dup_rows], ignore_index=True
)
trips_df.to_csv("trips.csv", index=False)

# --- Maintenance data ---
maint_types = [
    "tire_repair", "brake_adjustment",
    "battery_replacement", "chain_lubrication",
    "general_inspection"
]

records = []
for i in range(200):
    bike = np.random.choice(bike_ids)
    btype = np.random.choice(["classic", "electric"])
    mtype = np.random.choice(maint_types)
    cost = round(np.random.uniform(10, 150), 2)
    if mtype == "battery_replacement":
        cost = round(np.random.uniform(80, 250), 2)
        btype = "electric"

    records.append({
        "record_id": f"MR{5000 + i}",
        "bike_id": bike,
        "bike_type": btype,
        "date": (
            start_date
            + timedelta(days=np.random.randint(0, 365))
        ).strftime("%Y-%m-%d"),
        "maintenance_type": mtype,
        "cost": cost,
        "description": f"{mtype.replace('_', ' ').title()}"
                       f" for bike {bike}",
    })

maint_df = pd.DataFrame(records)
maint_df.loc[
    np.random.choice(200, 8, replace=False), "cost"
] = np.nan
maint_df.to_csv("maintenance.csv", index=False)

print("Generated: stations.csv, trips.csv, "
      "maintenance.csv")
\end{lstlisting}

% =========================
% 13. Submission
% =========================
\newpage
\section{Submission}

Submit:
\begin{enumerate}
  \item \textbf{GitHub repository URL} (must be accessible to the instructor)
  \item Repository must contain:
    \begin{itemize}
      \item All source code modules (\texttt{.py} files)
      \item \texttt{README.md} with project description, setup instructions, and usage guide
      \item \texttt{requirements.txt}
      \item Raw data files (\texttt{trips.csv}, \texttt{stations.csv}, \texttt{maintenance.csv})
      \item Cleaned datasets (\texttt{trips\_clean.csv}, \texttt{stations\_clean.csv})
      \item Summary report with answers to business questions (\texttt{summary\_report.txt})
      \item Visualization files (\texttt{output/figures/*.png})
    \end{itemize}
\end{enumerate}

\textbf{Execution:} Your project should run successfully via:
\begin{lstlisting}[language=bash]
$ pip install -r requirements.txt
$ python main.py
\end{lstlisting}

\textbf{Deadline:} announced separately. Late submissions lose 10\% per calendar day.

% =========================
% Footer note
% =========================
\vspace{10pt}
\hrule
\vspace{8pt}
\textcolor{muted}{\small Tip: Start with the data generator and models. Commit after every milestone. A working project with clean code is worth more than a feature-rich project that crashes.}

\end{document}
