\documentclass{beamer}
\usetheme{metropolis}
\usepackage{graphicx}
\usepackage{listings}
\usepackage{xcolor}


% Define custom colors
\definecolor{codegreen}{rgb}{0,0.6,0}
\definecolor{codegray}{rgb}{0.5,0.5,0.5}
\definecolor{codepurple}{rgb}{0.58,0,0.82}
\definecolor{backcolour}{rgb}{0.96,0.96,0.96}

\lstdefinestyle{mystyle}{
    backgroundcolor=\color{backcolour},
    commentstyle=\color{codegreen},
    keywordstyle=\color{blue},
    numberstyle=\tiny\color{codegray},
    stringstyle=\color{codepurple},
    basicstyle=\ttfamily\small,
    breakatwhitespace=false,
    breaklines=true,
    captionpos=b,
    % keepspaces=true,
    numbers=left,
    numbersep=5pt,
    showspaces=false,
    showstringspaces=false,
    showtabs=false,
    tabsize=2
}

\lstset{style=mystyle}


\title{Tag 1: Einrichten von Python-Umgebungen und Arbeiten mit Jupyter Notebooks}
\author{}
\date{\today}

\begin{document}

\maketitle

\section{Einführung}

\begin{frame}{Ziele des Tages}
  \begin{itemize}
    \item Verstehen, warum virtuelle Umgebungen nützlich sind.
    \item Erstellung und Verwaltung von virtuellen Umgebungen.
    \item Installation und Verwaltung von Python-Paketen.
    \item Einführung in Jupyter Notebooks.
    \item Arbeiten mit Jupyter-Zellen und Markdown.
  \end{itemize}
\end{frame}

\begin{frame}{Warum virtuelle Umgebungen?}
  \begin{itemize}
    \item Vermeidung von Versionskonflikten zwischen Projekten.
    \item Reproduzierbare Entwicklungsumgebung.
    \item Einfache Verwaltung von Abhängigkeiten.
  \end{itemize}
\end{frame}

\section{Virtuelle Umgebungen erstellen}

\begin{frame}[fragile]{Erstellen einer virtuellen Umgebung mit venv}
  \textbf{Schritte:}
  \begin{enumerate}
    \item Erstellen der Umgebung:
    \begin{lstlisting}[language=bash]
    python -m venv my_env
    \end{lstlisting}
    \item Aktivieren der Umgebung:
    \begin{lstlisting}[language=bash]
    source my_env/bin/activate  # Mac/Linux
    my_env\Scripts\activate  # Windows
    \end{lstlisting}
    \item Deaktivieren der Umgebung:
    \begin{lstlisting}[language=bash]
      deactivate
    \end{lstlisting}
  \end{enumerate}
\end{frame}

\begin{frame}[fragile]{Erstellen einer virtuellen Umgebung mit Conda}
  \textbf{Schritte:}
  \begin{enumerate}
    \item Erstellen der Umgebung:
    \begin{lstlisting}
    conda create --name my_env python=3.10
    \end{lstlisting}
    \item Aktivieren der Umgebung:
    \begin{lstlisting}
    conda activate my_env
    \end{lstlisting}
    \item Deaktivieren der Umgebung:
    \begin{lstlisting}
    conda deactivate
    \end{lstlisting}
  \end{enumerate}
\end{frame}

\section{Verwalten von Abhängigkeiten}

\begin{frame}[fragile]{Installieren von Paketen}
  \begin{itemize}
    \item Installation eines Pakets mit pip:
    \begin{lstlisting}
    pip install notebook
    \end{lstlisting}
    \item Anzeigen installierter Pakete:
    \begin{lstlisting}
    pip list
    \end{lstlisting}
    \item Erstellung einer Anforderungsliste:
    \begin{lstlisting}
    pip freeze > requirements.txt
    \end{lstlisting}
  \end{itemize}
\end{frame}

\section{Einführung in Jupyter Notebooks}

\begin{frame}{Was ist Jupyter Notebook?}
  \begin{itemize}
    \item Eine interaktive Umgebung für Python-Code.
    \item Unterstützt Code, Markdown, Diagramme und mehr.
    \item Perfekt für Datenanalyse und wissenschaftliches Rechnen.
  \end{itemize}
\end{frame}

\begin{frame}[fragile]{Starten von Jupyter Notebook}
  \begin{itemize}
    \item Sicherstellen, dass Jupyter installiert ist:
    \begin{lstlisting}
    pip install notebook
    \end{lstlisting}
    \item Starten von Jupyter:
    \begin{lstlisting}
    jupyter notebook
    \end{lstlisting}
  \end{itemize}
\end{frame}

\section{Arbeiten mit Jupyter-Zellen}

\begin{frame}[fragile]{Zelltypen in Jupyter}
  \begin{itemize}
    \item \textbf{Code-Zellen:} Python-Code ausführen
    \begin{lstlisting}
    print("Hallo, Jupyter!")
    \end{lstlisting}
    \item \textbf{Markdown-Zellen:} Dokumentation und Formatierung
    \begin{lstlisting}
    # Meine erste Markdown-Zelle
    **Fettgedruckt**, *kursiv*, `Code`
    \end{lstlisting}
  \end{itemize}
\end{frame}

\begin{frame}{Jupyter Shortcuts}
  \begin{itemize}
    \item `Shift + Enter` - Zelle ausführen
    \item `Esc + A/B` - Neue Zelle über/unter aktueller Zelle hinzufügen
    \item `Esc + D, D` - Zelle löschen
  \end{itemize}
\end{frame}

\section{Q\&A und Abschluss}

\begin{frame}{Zusammenfassung des Tages}
  \begin{itemize}
    \item Einrichtung und Nutzung virtueller Umgebungen.
    \item Installation und Verwaltung von Python-Paketen.
    \item Einführung in Jupyter Notebook.
    \item Arbeiten mit Jupyter-Zellen und Markdown.
  \end{itemize}
\end{frame}

\begin{frame}{Fragen und Diskussion}
  \centering
  \Huge{\textbf{Fragen?}}
\end{frame}

\end{document}
