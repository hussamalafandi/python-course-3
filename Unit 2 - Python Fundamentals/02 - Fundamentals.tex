\documentclass[aspectratio=169]{beamer}
\usetheme{metropolis}

\usepackage{graphicx}
\usepackage{listings}
\usepackage{xcolor}

% Define custom colors
\definecolor{codegreen}{rgb}{0,0.6,0}
\definecolor{codegray}{rgb}{0.5,0.5,0.5}
\definecolor{codepurple}{rgb}{0.58,0,0.82}
\definecolor{backcolour}{rgb}{0.96,0.96,0.96}

\lstdefinestyle{mystyle}{
    backgroundcolor=\color{backcolour},
    commentstyle=\color{codegreen},
    keywordstyle=\color{blue},
    numberstyle=\tiny\color{codegray},
    stringstyle=\color{codepurple},
    basicstyle=\ttfamily\small,
    breakatwhitespace=false,
    breaklines=true,
    captionpos=b,
    numbers=left,
    numbersep=5pt,
    showspaces=false,
    showstringspaces=false,
    showtabs=false,
    tabsize=2
}

\lstset{style=mystyle}

\title{Python 3 - Grundlagen und Ökosystem}
\subtitle{Unit 2: Python Fundamentals -- Syntax, Variablen \& Datentypen}
\author{Hussam Alafandi}
\date{\today}

\begin{document}

\maketitle

% =========================================================
% Unit Overview
% =========================================================
\section{Unit 2: Überblick}

\begin{frame}{Ziele der Unit}
  \begin{itemize}
    \item Python-Syntax und Einrückung (Indentation) verstehen.
    \item Variablen korrekt definieren und benennen.
    \item Grundlegende Datentypen sicher verwenden: \texttt{int}, \texttt{float}, \texttt{str}, \texttt{bool}.
    \item Typumwandlungen (Type Conversion) bewusst einsetzen.
    \item Operatoren anwenden: arithmetisch, Vergleich, logisch.
    \item Einfache Ein-/Ausgabe mit \texttt{input()} und \texttt{print()}.
    \item Häufige Anfängerfehler erkennen und Debugging-Basics nutzen.
  \end{itemize}
\end{frame}

\begin{frame}{Agenda}
  \begin{enumerate}
    \item Python-Syntax \& Indentation
    \item Variablen \& Namenskonventionen
    \item Datentypen in Python
    \item Typumwandlung
    \item Operatoren
    \item Ein-/Ausgabe (I/O)
    \item Debugging-Basics \& typische Fehler
    \item Praxisaufgaben
  \end{enumerate}
\end{frame}

% =========================================================
% Syntax & Indentation
% =========================================================
\section{Syntax \& Indentation}

\begin{frame}{Python-Syntax: Was ist besonders?}
  \begin{itemize}
    \item Python nutzt \textbf{Einrückung} statt geschweifter Klammern.
    \item Lesbarkeit ist ein Designziel der Sprache.
    \item Eine Zeile ist eine Anweisung (Statement).
    \item Kommentare beginnen mit \texttt{\#}.
  \end{itemize}
\end{frame}

\begin{frame}[fragile]{Indentation: Beispiel (korrekt)}
\begin{lstlisting}[language=Python]
x = 10

if x > 5:
    print("x ist größer als 5")
    print("Diese Zeile gehört noch zum if-Block")

print("Diese Zeile ist außerhalb des if-Blocks")
\end{lstlisting}
\end{frame}

\begin{frame}[fragile]{Indentation: Beispiel (Fehler)}
\begin{lstlisting}[language=Python]
x = 10

if x > 5:
print("x ist größer als 5")  # IndentationError
\end{lstlisting}
  \begin{itemize}
    \item Python erwartet nach \texttt{:} einen eingerückten Block.
    \item Typischer Fehler: \texttt{IndentationError}.
  \end{itemize}
\end{frame}

% =========================================================
% Variables & Naming
% =========================================================
\section{Variablen \& Namenskonventionen}

\begin{frame}{Variablen: Grundidee}
  \begin{itemize}
    \item Eine Variable ist ein Name, der auf einen Wert verweist.
    \item In Python hat der \textbf{Wert} einen Typ, nicht die Variable.
    \item Variablen können neu zugewiesen werden.
  \end{itemize}
\end{frame}

\begin{frame}[fragile]{Variablen: Beispiele}
\begin{lstlisting}[language=Python]
name = "Hussam"
age = 30
pi = 3.14159
is_admin = False

age = age + 1  # Re-Assignment
\end{lstlisting}
\end{frame}

\begin{frame}{Namenskonventionen (Best Practices)}
  \begin{itemize}
    \item Verwende sprechende Namen: \texttt{total\_price} statt \texttt{tp}.
    \item \textbf{snake\_case} für Variablen und Funktionen: \texttt{user\_age}.
    \item Keine reservierten Keywords: \texttt{if}, \texttt{for}, \texttt{class}, \dots
    \item Konstanten (Konvention): \texttt{MAX\_RETRIES = 3}
  \end{itemize}
\end{frame}

% =========================================================
% Data Types
% =========================================================
\section{Datentypen}

\begin{frame}{Built-in Datentypen (Überblick)}
  \begin{itemize}
    \item \texttt{int} -- Ganze Zahlen
    \item \texttt{float} -- Kommazahlen
    \item \texttt{str} -- Text (Strings)
    \item \texttt{bool} -- Wahr/Falsch
  \end{itemize}
\end{frame}

\begin{frame}[fragile]{Typen prüfen mit \texttt{type()}}
\begin{lstlisting}[language=Python]
x = 42
y = 3.14
z = "Python"
b = True

print(type(x))  # <class 'int'>
print(type(y))  # <class 'float'>
print(type(z))  # <class 'str'>
print(type(b))  # <class 'bool'>
\end{lstlisting}
\end{frame}

\begin{frame}[fragile]{Strings: Grundlagen}
\begin{lstlisting}[language=Python]
first_name = "Ada"
last_name = "Lovelace"

full_name = first_name + " " + last_name
print(full_name)

print(len(full_name))  # Länge des Strings
\end{lstlisting}
\end{frame}

\begin{frame}[fragile]{Booleans: Vergleich erzeugt True/False}
\begin{lstlisting}[language=Python]
x = 10
print(x > 5)   # True
print(x == 5)  # False
\end{lstlisting}
\end{frame}

% =========================================================
% Type Conversion
% =========================================================
\section{Typumwandlung (Type Conversion)}

\begin{frame}{Warum Typumwandlung?}
  \begin{itemize}
    \item \texttt{input()} liefert immer einen \texttt{str}.
    \item Für Berechnungen brauchen wir meist \texttt{int} oder \texttt{float}.
    \item Explizite Umwandlung macht Code verständlicher und sicherer.
  \end{itemize}
\end{frame}

\begin{frame}[fragile]{Typumwandlung: Beispiele}
\begin{lstlisting}[language=Python]
age_str = "30"
age_int = int(age_str)

pi_str = "3.14"
pi_float = float(pi_str)

print(age_int + 1)     # 31
print(pi_float * 2)    # 6.28
\end{lstlisting}
\end{frame}

\begin{frame}[fragile]{Typischer Fehler: String + Zahl}
\begin{lstlisting}[language=Python]
age = input("Alter: ")
print(age + 1)  # TypeError
\end{lstlisting}
  \begin{itemize}
    \item \texttt{age} ist ein \texttt{str} -- Addition mit \texttt{int} ist nicht erlaubt.
    \item Lösung: \texttt{int(age)} oder \texttt{float(age)}
  \end{itemize}
\end{frame}

% =========================================================
% Operators
% =========================================================
\section{Operatoren}

\begin{frame}{Arithmetische Operatoren}
  \begin{itemize}
    \item \texttt{+} Addition, \texttt{-} Subtraktion, \texttt{*} Multiplikation
    \item \texttt{/} Division (liefert \texttt{float})
    \item \texttt{//} Ganzzahl-Division
    \item \texttt{\%} Modulo (Rest)
    \item \texttt{**} Potenz
  \end{itemize}
\end{frame}

\begin{frame}[fragile]{Arithmetik: Beispiele}
\begin{lstlisting}[language=Python]
print(10 / 4)    # 2.5
print(10 // 4)   # 2
print(10 % 4)    # 2
print(2 ** 3)    # 8
\end{lstlisting}
\end{frame}

\begin{frame}{Vergleichsoperatoren}
  \begin{itemize}
    \item \texttt{==} gleich, \texttt{!=} ungleich
    \item \texttt{<, <=, >, >=}
    \item Ergebnis ist immer \texttt{bool}
  \end{itemize}
\end{frame}

\begin{frame}{Logische Operatoren}
  \begin{itemize}
    \item \texttt{and} -- beide Bedingungen müssen wahr sein
    \item \texttt{or} -- mindestens eine Bedingung wahr
    \item \texttt{not} -- negiert eine Bedingung
  \end{itemize}
\end{frame}

\begin{frame}[fragile]{Logik: Beispiele}
\begin{lstlisting}[language=Python]
age = 20
has_ticket = True

can_enter = (age >= 18) and has_ticket
print(can_enter)  # True
\end{lstlisting}
\end{frame}

% =========================================================
% Input / Output
% =========================================================
\section{Ein- und Ausgabe (I/O)}

\begin{frame}{\texttt{print()} und \texttt{input()}}
  \begin{itemize}
    \item \texttt{print()} gibt Werte aus.
    \item \texttt{input()} liest Text von der Konsole (immer \texttt{str}).
    \item Wir nutzen beides für kleine interaktive Programme.
  \end{itemize}
\end{frame}

\begin{frame}[fragile]{I/O: Interaktives Beispiel}
\begin{lstlisting}[language=Python]
name = input("Wie heißt du? ")
print("Hallo", name)

age = int(input("Wie alt bist du? "))
print("Nächstes Jahr bist du", age + 1)
\end{lstlisting}
\end{frame}

\begin{frame}[fragile]{String-Formatierung (f-strings)}
\begin{lstlisting}[language=Python]
name = "Ada"
age = 36
print(f"{name} ist {age} Jahre alt.")
\end{lstlisting}
\end{frame}

% =========================================================
% Debugging basics
% =========================================================
\section{Debugging \& typische Fehler}

\begin{frame}{Häufige Anfängerfehler}
  \begin{itemize}
    \item \texttt{SyntaxError}: Syntax falsch (z.\,B. Klammern, \texttt{:})
    \item \texttt{IndentationError}: Einrückung falsch
    \item \texttt{NameError}: Variable nicht definiert
    \item \texttt{TypeError}: falsche Typkombination (z.\,B. \texttt{"3" + 1})
    \item \texttt{ValueError}: Umwandlung nicht möglich (z.\,B. \texttt{int("abc")})
  \end{itemize}
\end{frame}

\begin{frame}[fragile]{Fehler lesen: Beispiel}
\begin{lstlisting}[language=Python]
age = int(input("Alter: "))
print("In 10 Jahren:", age + 10)
\end{lstlisting}
  \begin{itemize}
    \item Wenn der Nutzer \texttt{abc} eingibt $\rightarrow$ \texttt{ValueError}
    \item Debugging-Basics:
      \begin{itemize}
        \item Traceback von unten nach oben lesen
        \item Zeilennummer finden
        \item Annahmen prüfen (Typen, Werte)
      \end{itemize}
  \end{itemize}
\end{frame}

\begin{frame}[fragile]{Debugging-Basics: Kontrollausgaben}
\begin{lstlisting}[language=Python]
x = input("Gib eine Zahl ein: ")
print("DEBUG:", x, type(x))

n = int(x)
print("DEBUG:", n, type(n))
\end{lstlisting}
\end{frame}

% =========================================================
% Practice / Exercises
% =========================================================
\section{Praxis \& Aufgaben}

\begin{frame}{Praxis 1: Mini-Rechner}
  \begin{itemize}
    \item Schreibe ein Programm, das zwei Zahlen einliest.
    \item Berechne und drucke:
      \begin{itemize}
        \item Summe, Differenz, Produkt, Division
      \end{itemize}
    \item Bonus:
      \begin{itemize}
        \item Ganzzahl-Division und Modulo
      \end{itemize}
  \end{itemize}
\end{frame}

\begin{frame}{Praxis 2: String-Manipulation}
  \begin{itemize}
    \item Lies einen Namen ein und gib Folgendes aus:
      \begin{itemize}
        \item Name in Großbuchstaben
        \item Länge des Namens
        \item Initialen (Vorname/ Nachname, falls vorhanden)
      \end{itemize}
    \item Hinweis: Wir nutzen heute nur String-Methoden und \texttt{len()}.
  \end{itemize}
\end{frame}

\begin{frame}{Praxis 3: Einfache Input-Validierung}
  \begin{itemize}
    \item Lies das Alter ein.
    \item Wenn die Umwandlung fehlschlägt:
      \begin{itemize}
        \item Gib eine sinnvolle Fehlermeldung aus
      \end{itemize}
    \item Fokus: \textbf{ValueError verstehen} (Exception Handling folgt später ausführlich).
  \end{itemize}
\end{frame}

\begin{frame}[fragile]{Optional: Validierung mit \texttt{try/except}}
\begin{lstlisting}[language=Python]
age_str = input("Alter: ")

try:
    age = int(age_str)
    print(f"Du bist {age} Jahre alt.")
except ValueError:
    print("Bitte gib eine ganze Zahl ein.")
\end{lstlisting}
\end{frame}

\begin{frame}[standout]
  \huge Fragen?
\end{frame}

\end{document}
