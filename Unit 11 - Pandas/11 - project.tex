\documentclass[11pt,a4paper]{article}

% =========================
% Packages
% =========================
\usepackage[a4paper,margin=2.3cm,headheight=14pt]{geometry}
\usepackage[T1]{fontenc}
\usepackage[utf8]{inputenc} % remove if using lualatex/xelatex
\usepackage{lmodern}
\usepackage{microtype}
\usepackage{parskip}        % nicer paragraph spacing
\usepackage{enumitem}
\usepackage{booktabs}
\usepackage{tabularx}
\usepackage{xcolor}
\usepackage{hyperref}
\usepackage{fancyhdr}
\usepackage{titlesec}
\usepackage{listings}
\usepackage{graphicx}

% =========================
% Styling
% =========================
\definecolor{accent}{HTML}{0B3D91}
\definecolor{muted}{HTML}{6B7280}
\definecolor{backcolour}{HTML}{F6F7F9}
\definecolor{codegreen}{rgb}{0,0.55,0}
\definecolor{codegray}{rgb}{0.5,0.5,0.5}
\definecolor{codepurple}{rgb}{0.58,0,0.82}

\hypersetup{
  colorlinks=true,
  linkcolor=accent,
  urlcolor=accent,
  citecolor=accent
}

\titleformat{\section}{\large\bfseries\color{accent}}{}{0pt}{}
\titleformat{\subsection}{\normalsize\bfseries\color{accent}}{}{0pt}{}
\titleformat{\subsubsection}{\normalsize\bfseries}{}{0pt}{}

\setlist[itemize]{leftmargin=*, itemsep=0pt, topsep=2pt, parsep=0pt}
\setlist[enumerate]{leftmargin=*, itemsep=0pt, topsep=2pt, parsep=0pt}

\lstdefinestyle{mystyle}{
  backgroundcolor=\color{backcolour},
  commentstyle=\color{codegreen},
  keywordstyle=\color{accent},
  numberstyle=\tiny\color{codegray},
  stringstyle=\color{codepurple},
  basicstyle=\ttfamily\small,
  breaklines=true,
  numbers=left,
  numbersep=6pt,
  showstringspaces=false,
  tabsize=2,
  frame=single,
  rulecolor=\color{black!10}
}
\lstset{style=mystyle}

\pagestyle{fancy}
\fancyhf{}
\lhead{\textbf{Mini Project II}}
\rhead{\textcolor{muted}{Sales Analytics Platform}}
\cfoot{\thepage}

% =========================
% Document Meta
% =========================
\newcommand{\courseTitle}{Python Programming \& Data Analysis}
\newcommand{\unitTitle}{Mini Project II: Data Analysis \& OOP Integration}
\newcommand{\projectTitle}{Sales Analytics Platform}
\newcommand{\instructorName}{Hussam Alafandi}
\newcommand{\handoverDate}{\today}

\begin{document}

% =========================
% Header block
% =========================
{\Large \textbf{\courseTitle}}\\[-2pt]
{\large \textbf{\unitTitle}}\\[4pt]
{\normalsize \textcolor{muted}{Project Handout --- \projectTitle}}\\[10pt]

\begin{tabularx}{\textwidth}{@{}lX@{}}
\textbf{Date:} & \handoverDate \\
\textbf{Estimated workload:} & 6--10 hours (depending on optional extensions) \\
\textbf{Required libraries:} & \texttt{pandas}, \texttt{numpy}, \texttt{matplotlib} \\
\end{tabularx}

\vspace{10pt}
\hrule
\vspace{12pt}

% =========================
% 1. Overview
% =========================
\section{Project Overview}

In this project, you will build a \textbf{sales analytics platform} that consolidates everything learned in the course. You will design object-oriented data models, load and clean datasets, apply algorithms, generate reports, create visualizations, and manage your code with Git.

\textbf{This project integrates:}
\begin{itemize}
  \item Modular code structure, functions, multi-file organization (Units 1--5)
  \item Object-oriented design: classes, inheritance, design patterns (Units 7--8)
  \item Algorithms: sorting, searching, complexity analysis (Unit 9)
  \item NumPy for numerical computing (Unit 10)
  \item Pandas for data analysis, cleaning, and aggregation (Unit 11)
  \item Basic visualization with Matplotlib (Unit 12)
  \item Version control with Git and GitHub (Unit 13)
\end{itemize}

% =========================
% 2. Functional requirements
% =========================
\section{Functional Requirements}

\subsection{1) Object-Oriented Data Models}

Define classes to represent your domain:
\begin{itemize}
  \item \textbf{\texttt{Product}}: id, name, category, base price
  \item \textbf{\texttt{Customer}}: id, name, email, lifetime value
  \item \textbf{\texttt{Order}}: date, items, customer, amount, status
  \item \textbf{\texttt{SalesAnalyzer}}: orchestrates loading, cleaning, and analysis
\end{itemize}

Requirements:
\begin{itemize}
  \item Proper \texttt{\_\_init\_\_} constructors with input validation
  \item String representations (\texttt{\_\_str\_\_}, \texttt{\_\_repr\_\_})
  \item At least one inheritance hierarchy (e.g., base \texttt{Entity} class)
  \item Use of a design pattern (Factory or Strategy)
\end{itemize}

\subsection{2) Data Loading \& Cleaning}
\begin{itemize}
  \item Load CSV data into a Pandas DataFrame
  \item Inspect structure, types, and missing values (\texttt{info()}, \texttt{describe()})
  \item Handle missing values with a documented strategy
  \item Validate and convert types (dates, amounts)
  \item Remove duplicates or invalid entries
  \item Export a clean, analysis-ready dataset
\end{itemize}

\subsection{3) Algorithmic Analysis}

Implement \textbf{both} of the following:

\begin{itemize}
  \item \textbf{Sorting}:
    \begin{itemize}
      \item Implement your own sorting algorithm (e.g., quicksort, mergesort, or bubble sort)
      \item Also use built-in methods: Python's \texttt{sorted()}, Pandas \texttt{sort\_values()}, or NumPy \texttt{np.sort()}
      \item Compare execution time between your implementation and the built-in function using \texttt{timeit}
    \end{itemize}
  \item \textbf{Searching}:
    \begin{itemize}
      \item Implement your own search algorithm (e.g., linear search, binary search)
      \item Also use built-in methods: Python's \texttt{in} operator, Pandas \texttt{loc[]}/\texttt{query()}, or NumPy \texttt{np.where()}
      \item Compare execution time between your implementation and the built-in function
    \end{itemize}
\end{itemize}

\textbf{Optional Analysis:}
\begin{itemize}
  \item Document the \textbf{time complexity (Big-O)} for your implementations
  \item Include a brief comparison report discussing the performance differences
  \item Explain why built-in functions are typically faster (hint: optimized C implementations)
\end{itemize}

\subsection{4) Analytics \& Business Insights}

Using Pandas and NumPy, answer at least \textbf{8} of the following questions:
\begin{itemize}
  \item Total revenue, average order value (AOV), customer count
  \item Which product category is most profitable?
  \item Who are the top 10 customers by lifetime value?
  \item What is the repeat customer rate?
  \item Are there seasonal or monthly trends in sales?
  \item What is the average order size by category?
  \item What percentage of orders are cancelled vs. completed?
  \item Which orders are outliers (unusually large/small)?
  \item Customer segmentation by spending tier
  \item Revenue trends over time (monthly growth)
\end{itemize}

\subsection{5) Visualization}

Create at least \textbf{3 visualizations} using Matplotlib:
\begin{itemize}
  \item Bar chart (e.g., revenue by category)
  \item Line chart (e.g., monthly revenue trend)
  \item Histogram or boxplot (e.g., order value distribution)
\end{itemize}

Export visualizations as PNG files.

\subsection{6) Reporting \& Export}
\begin{itemize}
  \item Export cleaned dataset to CSV
  \item Generate a summary report (\texttt{.txt} or \texttt{.csv}) with key metrics
  \item Export top customers and products lists
\end{itemize}

% =========================
% 3. Data model
% =========================
\section{Data Model}

You will work with a CSV file: \texttt{sales\_data.csv}

Expected columns:
\begin{lstlisting}[language=bash]
order_id,customer_id,order_date,product_category,product_name,quantity,unit_price,order_amount,status
\end{lstlisting}

\textbf{Notes:}
\begin{itemize}
  \item Data may contain missing values, duplicates, or inconsistencies
  \item \texttt{order\_date} may need standardization to \texttt{YYYY-MM-DD}
  \item \texttt{order\_amount} may be stored as strings; convert to float
  \item \texttt{status}: \texttt{completed}, \texttt{cancelled}, or \texttt{pending}
\end{itemize}

% =========================
% 4. Code organization
% =========================
\section{Code Organization (Required)}

Use a multi-module project structure:

\begin{lstlisting}[language=bash]
sales_analytics/
  main.py              # Entry point / orchestration
  models.py            # OOP classes (Product, Customer, Order)
  analyzer.py          # SalesAnalyzer class with analysis methods
  algorithms.py        # Sorting, searching functions
  utils.py             # Validation, formatting helpers
  data/
    sales_data.csv     # Raw data
    sales_clean.csv    # Cleaned data
  output/
    summary_report.txt
    figures/           # Visualizations
\end{lstlisting}

\textbf{Module responsibilities:}

\begin{tabularx}{\textwidth}{@{}lX@{}}
\toprule
\textbf{File} & \textbf{Responsibility} \\
\midrule
\texttt{main.py} & Entry point; orchestrate loading, cleaning, analysis, reporting \\
\texttt{models.py} & OOP classes with validation and business logic \\
\texttt{analyzer.py} & Data analysis: groupby, filtering, metrics \\
\texttt{algorithms.py} & Sorting, searching, optimization functions \\
\texttt{utils.py} & Helpers for validation, formatting, dates \\
\bottomrule
\end{tabularx}

\textbf{Design principles:}
\begin{itemize}
  \item Keep business logic separate from I/O and printing
  \item Prefer functions that return values instead of printing inside
  \item Write docstrings for all functions and classes
\end{itemize}

% =========================
% 5. Git & GitHub Requirements
% =========================
\section{Version Control Requirements (Git \& GitHub)}

You \textbf{must} use Git for version control and host your project on GitHub.

\subsection{Repository Setup}
\begin{itemize}
  \item Create a new GitHub repository for this project
  \item Initialize with a \texttt{README.md} and \texttt{.gitignore} (Python template)
  \item Clone the repository locally and work from there
\end{itemize}

\subsection{Commit Requirements}
\begin{itemize}
  \item Make \textbf{at least 10 meaningful commits} throughout development
  \item Each commit should represent a logical unit of work
  \item Write clear, descriptive commit messages (e.g., ``Add Customer class with validation'')
\end{itemize}

\subsection{Branching (Recommended)}
\begin{itemize}
  \item Use feature branches for major components (e.g., \texttt{feature/oop-models}, \texttt{feature/visualization})
  \item Merge completed features into \texttt{main}
\end{itemize}

\subsection{Required Files in Repository}
\begin{itemize}
  \item \texttt{README.md} --- project description, setup instructions, usage
  \item \texttt{.gitignore} --- exclude \texttt{\_\_pycache\_\_}, \texttt{.venv}, IDE files
  \item \texttt{requirements.txt} --- list dependencies (\texttt{pandas}, \texttt{numpy}, \texttt{matplotlib})
\end{itemize}

% =========================
% 6. Milestones
% =========================
\section{Milestones}

\begin{enumerate}
  \item \textbf{Milestone 1 (Setup)}: Create GitHub repo, project structure, load raw data
  \item \textbf{Milestone 2 (OOP)}: Implement all required classes with validation
  \item \textbf{Milestone 3 (Cleaning)}: Clean data, handle missing values, export clean CSV
  \item \textbf{Milestone 4 (Analysis)}: Implement analytics, answer 8+ business questions
  \item \textbf{Milestone 5 (Algorithms)}: Implement sorting/searching, compare with built-ins
  \item \textbf{Milestone 6 (Visualization)}: Create 3+ charts, export as PNG
  \item \textbf{Milestone 7 (Finalize)}: Generate reports, update README, final commit
\end{enumerate}

% =========================
% 7. Quality checklist
% =========================
\newpage
\section{Quality Checklist}

Before submission, verify:

\textbf{Code Quality:}
\begin{itemize}
  \item Code is organized into multiple modules
  \item Functions are modular with meaningful names
  \item No crashes on invalid input
  \item Comments explain non-obvious logic
\end{itemize}

\textbf{OOP Design:}
\begin{itemize}
  \item Classes have clear responsibilities
  \item Constructors validate input
  \item At least one inheritance hierarchy
  \item \texttt{\_\_str\_\_} and \texttt{\_\_repr\_\_} implemented
\end{itemize}

\textbf{Data Analysis:}
\begin{itemize}
  \item Data properly loaded, inspected, and cleaned
  \item NumPy/Pandas used appropriately
  \item All 8+ business questions answered
\end{itemize}

\textbf{Algorithms:}
\begin{itemize}
  \item Own sorting algorithm implemented
  \item Own searching algorithm implemented
  \item Built-in functions also used for comparison
  \item Performance comparison with \texttt{timeit} included
  \item Big-O complexity documented
\end{itemize}

\textbf{Visualization:}
\begin{itemize}
  \item At least 3 charts created
  \item Charts are labeled and readable
  \item Exported as PNG files
\end{itemize}

\textbf{Git \& GitHub:}
\begin{itemize}
  \item Repository is public (or shared with instructor)
  \item At least 10 meaningful commits
  \item Clear commit messages
  \item README.md with instructions
  \item \texttt{requirements.txt} included
\end{itemize}

% =========================
% 8. Synthetic data template
% =========================
\section{Appendix: Synthetic Data Generator}

If you need a dataset, use this code:

\begin{lstlisting}[language=Python]
import pandas as pd
import numpy as np
from datetime import datetime, timedelta

np.random.seed(42)
n_orders = 200
categories = ["Electronics", "Clothing", "Home & Garden", "Sports", "Books"]
products = {
    "Electronics": ["Laptop", "Phone", "Tablet", "Headphones"],
    "Clothing": ["T-Shirt", "Jeans", "Jacket", "Shoes"],
    "Home & Garden": ["Lamp", "Plant", "Cushion", "Rug"],
    "Sports": ["Yoga Mat", "Dumbbell", "Running Shoes", "Bike"],
    "Books": ["Fiction", "Science", "History", "Art"]
}

orders = []
start_date = datetime(2023, 1, 1)

for i in range(n_orders):
    category = np.random.choice(categories)
    product = np.random.choice(products[category])
    qty = np.random.randint(1, 5)
    unit_price = np.random.uniform(10, 500)
    amount = qty * unit_price
    status = np.random.choice(
        ["completed", "pending", "cancelled", np.nan],
        p=[0.7, 0.15, 0.1, 0.05]
    )
    orders.append({
        "order_id": f"ORD{1000+i}",
        "customer_id": f"CUST{np.random.randint(1, 50)}",
        "order_date": start_date + timedelta(days=np.random.randint(0, 365)),
        "product_category": category,
        "product_name": product,
        "quantity": qty,
        "unit_price": round(unit_price, 2),
        "order_amount": round(amount, 2),
        "status": status
    })

df = pd.DataFrame(orders)
df.to_csv("sales_data.csv", index=False)
\end{lstlisting}

% =========================
% 9. Submission
% =========================
\section{Submission}

Submit:
\begin{enumerate}
  \item \textbf{GitHub repository URL} (must be accessible)
  \item Repository must contain:
    \begin{itemize}
      \item All source code (\texttt{.py} files)
      \item \texttt{README.md} with setup and run instructions
      \item \texttt{requirements.txt}
      \item Cleaned dataset (\texttt{sales\_clean.csv})
      \item Summary report with answers to business questions
      \item Visualization files (PNG)
    \end{itemize}
\end{enumerate}

\textbf{Execution:} Your project should run via:
\begin{lstlisting}[language=bash]
$ pip install -r requirements.txt
$ python main.py
\end{lstlisting}

% =========================
% Footer note
% =========================
\vspace{10pt}
\hrule
\vspace{8pt}
\textcolor{muted}{\small Tip: Commit often. Each milestone should have at least one commit.}

\end{document}

\end{document}
