\documentclass[a4paper,12pt]{article}

\usepackage[T1]{fontenc}
\usepackage[utf8]{inputenc}
\usepackage[ngerman]{babel}
\usepackage{amsmath,amssymb}
\usepackage{enumitem}
\usepackage{listings}
\usepackage{xcolor}
\usepackage{geometry}
\usepackage{fancyhdr}
\usepackage[hidelinks]{hyperref}

\geometry{left=2.5cm, right=2.5cm, top=3.5cm, bottom=3.0cm}

% Define colors and style for code listings
\definecolor{codegreen}{rgb}{0,0.6,0}
\definecolor{codegray}{rgb}{0.5,0.5,0.5}
\definecolor{codepurple}{rgb}{0.58,0,0.82}
\definecolor{backcolour}{rgb}{0.96,0.96,0.96}

\lstdefinestyle{mystyle}{
    backgroundcolor=\color{backcolour},
    commentstyle=\color{codegreen},
    keywordstyle=\color{blue},
    numberstyle=\tiny\color{codegray},
    stringstyle=\color{codepurple},
    basicstyle=\ttfamily\footnotesize,
    breakatwhitespace=false,
    breaklines=true,
    captionpos=b,
    numbers=left,
    numbersep=5pt,
    showspaces=false,
    showstringspaces=false,
    showtabs=false,
    tabsize=2
}
\lstset{style=mystyle}

% Fancy header and footer settings
\pagestyle{fancy}
\fancyhead[L]{Python Programmierung III}
\fancyhead[R]{\today}
\fancyfoot[C]{}
\fancyfoot[R]{\thepage}
\setlength{\headheight}{14.5pt}

\title{Vorbereitungsklausur\\
     Python Programmierung III}
% \author{[Ihr Name oder Kursname]}
\date{\today}

\begin{document}

\maketitle

% --- Student Information Fields ---
\thispagestyle{empty}
\vspace{1cm}
\begin{center}
    \textbf{Name:} \rule{8cm}{0.4pt} \\[0.5cm]
    \textbf{Vorname:} \rule{8cm}{0.4pt} \\[0.5cm]
    \textbf{Unterschrift:} \rule{8cm}{0.4pt}
\end{center}


\vfil

\begin{center}
    \begin{tabular}{|c|c|c|}
        \hline
        \textbf{Aufgabe}                                & \textbf{Max. Punkte} & \textbf{Erreichte Punkte} \\ \hline
    \ref{sec:python_environments}   & 4 &                           \\ \hline
    \ref{sec:oop}                   & 6 &                           \\ \hline
    \ref{sec:algorithms}            & 6 &                           \\ \hline
    \ref{sec:numpy}                 & 6 &                           \\ \hline
    \ref{sec:pandas}                & 7 &                           \\ \hline
    \textbf{Gesamt}                 & 29 &                          \\ \hline
    \end{tabular}
\end{center}


%%%%%%%%%%%%%%%%%%%%%%%%%%%%%%%%%%%%%%%%%%%%%%%%%%%%%%%%%%%%%%%%%%%%%%
\clearpage
\section{Python Umgebungen (4 Punkte)}\label{sec:python_environments}
\begin{enumerate}
    \item Erklären Sie mit eigenen Worten, \textbf{was} ein virtuelles Environment ist und \textbf{warum} es nützlich ist.\ \textbf{(2 Punkte)}

          \vfil

    \item Welche Befehle verwenden Sie, um ein virtuelles Environment zu \textbf{erstellen} und zu \textbf{aktivieren} mit \texttt{venv} in Python? \textbf{(2 Punkte)}
\end{enumerate}


%%%%%%%%%%%%%%%%%%%%%%%%%%%%%%%%%%%%%%%%%%%%%%%%%%%%%%%%%%%%%%%%%%%%%%
\clearpage
\section{Objektorientierte Programmierung (OOP) (6 Punkte)}\label{sec:oop}

\begin{enumerate}
    \item Erklären Sie mit eigenen Worten, \textbf{was} sind abstrakte Klassen und \textbf{warum} sind sie nützlich? Geben Sie ein Beispiel für eine abstrakte Klasse in Python. (2 Punkte)

          \vfill

    \item Der folgende Code definiert eine Klasse \texttt{Person} mit einem privaten Attribut \texttt{\_\_password}. Ergänzen Sie die Methode \texttt{change\_password}, sodass sie das Passwort nur dann ändert, wenn das alte Passwort korrekt eingegeben wurde. Falls das alte Passwort nicht stimmt, soll eine Fehlermeldung ausgegeben werden. (2 Punkte)

          \vspace{0.3cm}
          \begin{lstlisting}[language=Python]
class Person:
  def __init__(self, name, password):
      self.name = name
      self.__password = password  # Privates Attribut
  
  def change_password(self, old_password, new_password):

  








    
.
    \end{lstlisting}



          \newpage
    \item Erklären Sie den Zweck der Funktion \texttt{super()} in einer Unterklasse? (1 Punkt)

          \vfil


    \item Nennen Sie zwei Entwurfmustern. (1 Punkt)
\end{enumerate}

\newpage

%%%%%%%%%%%%%%%%%%%%%%%%%%%%%%%%%%%%%%%%%%%%%%%%%%%%%%%%%%%%%%%%%%%%%%
\section{Algorithmen, Effizienz und Rekursion (6 Punkte)}\label{sec:algorithms}

\begin{enumerate}
    \item Erklären Sie den Unterschied zwischen linearer Suche und binärer Suche. Geben Sie die Zeitkomplexität für lineare Suche. (2 Punkte)
          \vfil

    \item Der folgende Code soll die Summe aller natürlichen Zahlen von \( 1 \) bis \( n \) rekursiv berechnen. Ergänzen Sie die fehlende Zeile für die Basisfallbedingung. (2 Punkt)

          \vspace{0.3cm}
          \begin{lstlisting}[language=Python]
def summe(n):





  return n + summe(n-1)

print(summe(5))  # Erwartete Ausgabe: 15
    \end{lstlisting}

          \textbf{Hinweis:} Der Basisfall ist erforderlich, um eine unendliche Rekursion zu vermeiden.

          \newpage
    \item Der folgende Python-Code implementiert den Sortieralgorithmus \texttt{Insertion Sort}. Analysieren Sie den Code und geben Sie die Zeitkomplexität des Algorithmus an. (2 Punkte)

          \vspace{0.3cm}
          \begin{lstlisting}[language=Python]
def insertion_sort(arr):
    for i in range(1, len(arr)):
        key = arr[i]
        j = i - 1
        while j >= 0 and arr[j] > key:
            arr[j + 1] = arr[j]
            j -= 1
        arr[j + 1] = key
    return arr
      \end{lstlisting}

\end{enumerate}

\newpage

%%%%%%%%%%%%%%%%%%%%%%%%%%%%%%%%%%%%%%%%%%%%%%%%%%%%%%%%%%%%%%%%%%%%%%
\section{NumPy (6 Punkte)}\label{sec:numpy}

\begin{enumerate}
    \item Erklären Sie den Unterschied zwischen einem NumPy-Array und einer Python-Liste. Welche Vorteile bieten NumPy-Arrays gegenüber Python-Listen? (2 Punkte)

          \vfil

    \item Betrachten Sie folgenden Code. Geben Sie die erwartete Ausgabe an und begründen Sie Ihre Antwort. (1 Punkt)
          \vspace{0.3cm}
          \begin{lstlisting}[language=Python]
import numpy as np

a = np.array([1, 2, 3])
b = np.array([[10], [20], [30]])

result = a + b
print(result)
  \end{lstlisting}


          \newpage
    \item Ergänzen Sie den folgenden Code, um eine Matrixmultiplikation zwischen den beiden NumPy-Arrays \texttt{a} und \texttt{b} durchzuführen. (1 Punkte)
          \vspace{0.3cm}
          \begin{lstlisting}[language=Python]
import numpy as np

a = np.array([[1, 2], [3, 4]])
b = np.array([[5, 6], [7, 8]])

result = _____________________
print(result)
    \end{lstlisting}


          \vfil
    \item Was versteht man unter Broadcasting in NumPy? Erklären Sie anhand eines Beispiels. (2 Punkte)
\end{enumerate}

\newpage

%%%%%%%%%%%%%%%%%%%%%%%%%%%%%%%%%%%%%%%%%%%%%%%%%%%%%%%%%%%%%%%%%%%%%%
\section{Pandas (7 Punkte)}\label{sec:pandas}

\begin{enumerate}
    \item Erklären Sie den Unterschied zwischen einer Series und einem DataFrame in Pandas. Wann würden Sie eine Series verwenden und wann ein DataFrame? (2 Punkte)
          \vfil

    \item Betrachten Sie den folgenden Code. Was wird ausgegeben, wenn der Code ausgeführt wird? (1 Punkt)
          \vspace{0.3cm}
          \begin{lstlisting}[language=Python]
import pandas as pd

data = {'Name': ['Anna', 'Bernd', 'Clara', 'David'],
        'Alter': [28, 34, 29, 42],
        'Stadt': ['Berlin', 'München', 'Hamburg', 'Köln']}

df = pd.DataFrame(data)
print(df['Alter'] > 30)
  \end{lstlisting}
          \newpage

    \item Vervollständigen Sie den folgenden Code, der eine CSV-Datei einliest und alle Zeilen ausgibt, in denen der \texttt{total\_bill} größer als 20 ist. (1 Punkt)
          \vspace{0.3cm}
          \begin{lstlisting}[language=Python]
df = pd.read_csv("tips.csv")
df_filtered = df[ ___________________________ ]
print(df_filtered.head())
  \end{lstlisting}

          \vfil
    \item Beschreiben Sie, wie man in Pandas fehlende Werte erkennt und behandelt. Was ist der Unterschied zwischen \texttt{fillna()} und \texttt{dropna()}? (2 Punkte)
          \vfil

    \item Ergänzen Sie den folgenden Code, um eine neue Spalte \texttt{Tip\_Percentage} zu erstellen, die den Prozentsatz des Trinkgelds am Gesamtbetrag berechnet. (1 Punkt)
          \vspace{0.3cm}
          \begin{lstlisting}[language=Python]
df = pd.read_csv("tips.csv")
df["Tip_Percentage"] = ______________________________
print(df[["total_bill", "tip", "Tip_Percentage"]].head())
  \end{lstlisting}
\end{enumerate}


\end{document}
