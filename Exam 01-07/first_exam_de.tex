\documentclass[11pt,a4paper]{article}

% Packages
\usepackage[utf8]{inputenc}
\usepackage[T1]{fontenc}
\usepackage[ngerman]{babel}
\usepackage[margin=2.5cm]{geometry}
\usepackage{amsmath,amssymb}
\usepackage{enumitem}
\usepackage{listings}
\usepackage{xcolor}
\usepackage{tikz}
\usetikzlibrary{shapes.geometric, arrows, positioning, calc}
\usepackage{fancyhdr}
\usepackage{lastpage}
\usepackage{array}
\usepackage{tabularx}
\usepackage{booktabs}

% Code listing style
\definecolor{codebg}{RGB}{248,248,248}
\definecolor{codeframe}{RGB}{200,200,200}
\definecolor{codekw}{RGB}{0,0,180}
\definecolor{codecomment}{RGB}{0,128,0}
\definecolor{codestring}{RGB}{163,21,21}

\lstdefinestyle{pythonstyle}{
    language=Python,
    basicstyle=\ttfamily\small,
    keywordstyle=\color{codekw}\bfseries,
    commentstyle=\color{codecomment}\itshape,
    stringstyle=\color{codestring},
    backgroundcolor=\color{codebg},
    frame=single,
    rulecolor=\color{codeframe},
    numbers=left,
    numberstyle=\tiny\color{gray},
    numbersep=8pt,
    showstringspaces=false,
    showlines=true,
    breaklines=true,
    tabsize=4,
    xleftmargin=15pt,
    framexleftmargin=15pt,
}

\lstset{style=pythonstyle}

% Flowchart styles
\tikzstyle{startstop} = [rectangle, rounded corners, minimum width=2.5cm, minimum height=0.8cm, text centered, draw=black, fill=red!20]
\tikzstyle{process} = [rectangle, minimum width=2.5cm, minimum height=0.8cm, text centered, draw=black, fill=blue!15]
\tikzstyle{decision} = [diamond, aspect=2, minimum width=2cm, minimum height=0.8cm, text centered, draw=black, fill=green!15]
\tikzstyle{io} = [trapezium, trapezium left angle=70, trapezium right angle=110, minimum width=2cm, minimum height=0.8cm, text centered, draw=black, fill=orange!15]
\tikzstyle{arrow} = [thick,->,>=stealth]

% Header and footer
\pagestyle{fancy}
\fancyhf{}
\lhead{Python-Programmierung -- Zwischenprüfung}
\rhead{}
\cfoot{Seite \thepage\ von \pageref{LastPage}}

% Points box
\newcommand{\pointsbox}[1]{\hfill\fbox{\textbf{#1 Pkt.}}}

\begin{document}

% Title section
\begin{center}
  {\Large\bfseries Python-Programmierung}\\[0.3cm]
  {\large Zwischenprüfung}\\[0.3cm]
  {\normalsize 23.01.2026}\\[0.5cm]

  \vfil
  \textbf{Name:} \hrulefill
  \vfil

\end{center}

\vspace{0.5cm}

\noindent\textbf{Hinweise:}
\begin{itemize}[noitemsep]
  \item Bearbeitungszeit: \textbf{2 Stunden}
  \item Gesamtpunktzahl: \textbf{60 Punkte}
  \item Keine Hilfsmittel erlaubt (keine Notizen, keine elektronischen Geräte)
\end{itemize}

\vspace{0.3cm}

\noindent\textbf{Punkteübersicht:}
\begin{center}
  \begin{tabular}{|c|c|c|}
    \hline
    \textbf{Aufgabe} & \textbf{Max. Punkte} & \textbf{Erreicht} \\
    \hline
    1                 & 4                   &                \\
    \hline
    2                 & 6                   &                \\
    \hline
    3                 & 5                   &                \\
    \hline
    4                 & 7                   &                \\
    \hline
    5                 & 8                   &                \\
    \hline
    6                 & 6                   &                \\
    \hline
    7                 & 9                   &                \\
    \hline
    8                 & 15                  &                \\
    \hline
    \textbf{Gesamt}   & \textbf{60}         &                \\
    \hline
  \end{tabular}
\end{center}

\newpage

%==============================================================================
% AUFGABE 1: Multiple Choice (Grundlagen & Typen)
%==============================================================================
\section*{Aufgabe 1: Multiple Choice \pointsbox{4}}

\textit{Kreisen Sie die richtige Antwort ein. Jede richtige Antwort ist 1 Punkt wert.}

\vspace{0.4cm}

\begin{enumerate}[label=\textbf{1.\arabic*}]
  \item Was ist die Ausgabe des folgenden Ausdrucks?
        \begin{lstlisting}[numbers=none]
result = 17 // 5
print(result)
    \end{lstlisting}
        \begin{enumerate}[label=\Alph*)]
          \item \texttt{3.4}
          \item \texttt{3}
          \item \texttt{2}
          \item \texttt{17}
        \end{enumerate}

        \vspace{0.5cm}

  \item Welche der folgenden Datenstrukturen ist in Python \textbf{veränderbar} (mutable)?
        \begin{enumerate}[label=\Alph*)]
          \item \texttt{tuple}
          \item \texttt{str}
          \item \texttt{list}
          \item \texttt{int}
        \end{enumerate}

        \vspace{0.5cm}

  \item Was gibt \texttt{type(3.14)} zurück?
        \begin{enumerate}[label=\Alph*)]
          \item \texttt{<class 'int'>}
          \item \texttt{<class 'float'>}
          \item \texttt{<class 'str'>}
          \item \texttt{<class 'number'>}
        \end{enumerate}

        \vspace{0.5cm}

  \item Was repräsentiert das Schlüsselwort \texttt{self} in einer Klassenmethode in Python?
        \begin{enumerate}[label=\Alph*)]
          \item Die Klasse selbst
          \item Die Elternklasse
          \item Die aktuelle Instanz/das aktuelle Objekt
          \item Eine globale Variable
        \end{enumerate}
\end{enumerate}

\newpage

%==============================================================================
% AUFGABE 2: Ausgabe vorhersagen
%==============================================================================
\section*{Aufgabe 2: Ausgabe vorhersagen \pointsbox{6}}

\textit{Schreiben Sie für jeden Code-Ausschnitt genau auf, was ausgegeben wird. Achten Sie auf Abstände und Format.}

\vspace{0.4cm}

\begin{enumerate}[label=\textbf{2.\arabic*}]
  \item \textbf{(1 Pkt.)}
        \begin{lstlisting}
numbers = [10, 20, 30, 40, 50]
print(numbers[1:4])
print(numbers[-2])
    \end{lstlisting}
        \textbf{Ausgabe:}
        \vfil

  \item \textbf{(2 Pkt.)}
        \begin{lstlisting}
x = 5
y = 3
x, y = y, x + y
print(f"x={x}, y={y}")
    \end{lstlisting}
        \textbf{Ausgabe:}


        \newpage
  \item \textbf{(1 Pkt.)}
        \begin{lstlisting}
data = {"a": 1, "b": 2, "c": 3}
for key in data:
    if data[key] > 1:
        print(key, end=" ")
    \end{lstlisting}
        \textbf{Ausgabe:}
        \vfil

  \item \textbf{(2 Pkt.)}
        \begin{lstlisting}
def process(items):
    items.append(4)
    items = [100]
    return items

original = [1, 2, 3]
result = process(original)
print(original)
print(result)
    \end{lstlisting}
        \textbf{Ausgabe:}
        \vfil
\end{enumerate}

\newpage

%==============================================================================
% AUFGABE 3: Fehler finden
%==============================================================================
\section*{Aufgabe 3: Fehler finden und korrigieren \pointsbox{5}}

\textit{Jeder Code-Ausschnitt enthält einen oder mehrere Fehler. Identifizieren Sie jeden Fehler und schreiben Sie die korrigierte(n) Zeile(n). Erklären Sie kurz, was falsch war.}

\vspace{0.4cm}

\begin{enumerate}[label=\textbf{3.\arabic*}]
  \item \textbf{(1 Pkt.)} Der folgende Code soll ,,Hello, World!'' ausgeben, enthält aber einen Fehler:
        \begin{lstlisting}
message = "Hello, World!
print(message)
    \end{lstlisting}
        \textbf{Fehler:}
        \vspace{1cm}

        \textbf{Korrektur:}
        \vspace{1.5cm}

  \item \textbf{(2 Pkt.)} Die folgende Funktion soll die Summe aller Zahlen in einer Liste zurückgeben:
        \begin{lstlisting}
def sum_list(numbers)
    total = 0
    for num in numbers:
    total += num
    return total
    \end{lstlisting}
        \textbf{Fehler (alle auflisten):}
        \vspace{2cm}

        \textbf{Korrigierter Code:}
        \vspace{3cm}

  \item \textbf{(2 Pkt.)} Die folgende Klassendefinition enthält einen Fehler:
        \begin{lstlisting}
class Dog:
    def __init__(name, breed):
        self.name = name
        self.breed = breed

dog1 = Dog("Buddy", "Labrador")
    \end{lstlisting}
        \textbf{Fehler:}
        \vspace{1cm}

        \textbf{Korrektur:}
        \vspace{1.5cm}
\end{enumerate}

\newpage

%==============================================================================
% AUFGABE 4: Flussdiagramm zu Code
%==============================================================================
\section*{Aufgabe 4: Flussdiagramm zu Python-Code \pointsbox{7}}

\textit{Studieren Sie das folgende Flussdiagramm und schreiben Sie den entsprechenden Python-Code. Das Flussdiagramm nimmt eine positive ganze Zahl und erzeugt ein Ergebnis basierend auf dem Extrahieren und Verarbeiten ihrer Ziffern.}

\vspace{0.4cm}

\begin{center}
  \begin{tikzpicture}[node distance=1.3cm]
    \node (start) [startstop] {Start};
    \node (input) [io, below of=start] {Eingabe: n};
    \node (init) [process, below of=input] {result = 0};
    \node (check) [decision, below of=init, yshift=-0.6cm] {n $>$ 0?};
    \node (getdigit) [process, right of=check, xshift=3cm] {digit = n \% 10};
    \node (build) [process, below of=getdigit] {result = result * 10 + digit};
    \node (reduce) [process, below of=build] {n = n // 10};
    \node (output) [io, below of=check, yshift=-2.5cm] {Ausgabe: result};
    \node (stop) [startstop, below of=output] {Stop};

    \draw [arrow] (start) -- (input);
    \draw [arrow] (input) -- (init);
    \draw [arrow] (init) -- (check);
    \draw [arrow] (check) -- node[anchor=south] {Ja} (getdigit);
    \draw [arrow] (getdigit) -- (build);
    \draw [arrow] (build) -- (reduce);
    % Loop back: go right, then up, then left to join the line between init and check
    \coordinate (loopback) at ($(init.south)!0.5!(check.north)$);
    \draw [arrow] (reduce.east) -- +(2,0) |- (loopback);
    \draw [arrow] (check) -- node[anchor=east] {Nein} (output);
    \draw [arrow] (output) -- (stop);
  \end{tikzpicture}
\end{center}

\vspace{0.3cm}
\textbf{Hinweis:} Verfolgen Sie das Flussdiagramm mit \texttt{n = 123}, um zu verstehen, was es tut, bevor Sie den Code schreiben.

\vspace{0.2cm}
\textbf{Erinnerung:} \texttt{\%} gibt den Rest, \texttt{//} gibt den ganzzahligen Quotienten.\\
\textit{Beispiel:} \texttt{47 \% 10 = 7} \quad und \quad \texttt{47 // 10 = 4}

\vspace{0.3cm}
\textbf{Schreiben Sie den Python-Code, der dieses Flussdiagramm implementiert, auf der nächsten Seite.}

\newpage

\textbf{Aufgabe 4 -- Antwortbereich:}

\vfill

\newpage

%==============================================================================
% AUFGABE 5: Code zu Flussdiagramm
%==============================================================================
\section*{Aufgabe 5: Python-Code zu Flussdiagramm \pointsbox{8}}

\textit{Zeichnen Sie für den folgenden Python-Code ein Flussdiagramm, das seine Logik darstellt. Verwenden Sie die richtigen Flussdiagrammsymbole (Oval für Start/Stop, Rechteck für Verarbeitung, Raute für Entscheidung, Parallelogramm für Ein-/Ausgabe).}

\vspace{0.4cm}

\begin{lstlisting}
def analyze_grades(grades):
    if len(grades) == 0:
        return "No grades", 0, 0
    
    total = 0
    passed = 0
    
    for grade in grades:
        total = total + grade
        if grade >= 50:
            passed = passed + 1
    
    average = total / len(grades)
    
    if average >= 70:
        status = "Excellent"
    elif average >= 50:
        status = "Passed"
    else:
        status = "Failed"
    
    return status, average, passed
\end{lstlisting}

\vspace{0.3cm}
\textbf{Zeichnen Sie Ihr Flussdiagramm auf der nächsten Seite.}

\newpage

\textbf{Aufgabe 5 -- Antwortbereich:}

\vfill

\newpage

%==============================================================================
% AUFGABE 6: Datenstrukturen
%==============================================================================
\section*{Aufgabe 6: Datenstrukturen \pointsbox{6}}

\begin{enumerate}[label=\textbf{6.\arabic*}]
  \item \textbf{(2 Pkt.)} Schreiben Sie für das folgende Dictionary Python-Code, um:
        \begin{lstlisting}[numbers=none]
students = {
    "Alice": 85,
    "Bob": 72,
    "Charlie": 91,
    "Diana": 68
}
    \end{lstlisting}

        \begin{enumerate}[label=\alph*)]
          \item Einen neuen Studenten ,,Eve'' mit Note 88 hinzuzufügen:
                \vspace{1.5cm}

          \item Alle Studenten auszugeben, die über 80 Punkte erreicht haben:
                \vspace{2.5cm}
        \end{enumerate}

  \item \textbf{(2 Pkt.)} Was ist der Unterschied zwischen einer \textbf{Liste} und einem \textbf{Tupel}? Nennen Sie je einen praktischen Anwendungsfall.
        \vspace{3.5cm}

  \item \textbf{(2 Pkt.)} Was wird der Inhalt von \texttt{my\_set} nach Ausführung des folgenden Codes sein?
        \begin{lstlisting}
my_set = {1, 2, 3, 2, 4, 1, 5, 3}
my_set.add(6)
my_set.add(2)
my_set.remove(4)
    \end{lstlisting}
        \textbf{Antwort:}
        \vspace{1.5cm}
\end{enumerate}

\newpage

%==============================================================================
% AUFGABE 7: Funktionen
%==============================================================================
\section*{Aufgabe 7: Funktionen \pointsbox{9}}

\begin{enumerate}[label=\textbf{7.\arabic*}]
  \item \textbf{(3 Pkt.)} Vervollständigen Sie die folgende Funktion, die eine Liste von Zahlen nimmt und eine neue Liste mit nur den geraden Zahlen zurückgibt:
        \begin{lstlisting}
def filter_even(numbers):
    # Ihr Code hier
 
 
 
 
 
 
 
 
 
 
 
 
 
 

    \end{lstlisting}
\vfil
  \item \textbf{(3 Pkt.)} Erklären Sie den Unterschied zwischen diesen beiden Funktionen. Was wird jeweils ausgegeben, wenn sie aufgerufen werden?
        \begin{lstlisting}
def func_a(x):
    print(x * 2)

def func_b(x):
    return x * 2

result_a = func_a(5)
result_b = func_b(5)
print(f"result_a: {result_a}")
print(f"result_b: {result_b}")
    \end{lstlisting}
        \textbf{Erklärung und Ausgabe:}
        \vfil

        \newpage
  \item \textbf{(3 Pkt.)} Schreiben Sie eine Funktion namens \texttt{count\_vowels}, die einen String nimmt und die Anzahl der Vokale (a, e, i, o, u) darin zurückgibt. Die Funktion soll sowohl für Groß- als auch Kleinbuchstaben funktionieren.

        \textbf{Beispiel:} \texttt{count\_vowels("Hello World")} soll \texttt{3} zurückgeben

        \vspace{7cm}
\end{enumerate}

\newpage

%==============================================================================
% AUFGABE 8: Objektorientierte Programmierung
%==============================================================================
\section*{Aufgabe 8: Objektorientierte Programmierung \pointsbox{15}}

\begin{enumerate}[label=\textbf{8.\arabic*}]
  \item \textbf{(4 Pkt.)} Was ist die Ausgabe des folgenden Codes? Erklären Sie warum.
        \begin{lstlisting}
class ProgressTracker:
    def __init__(self, total_steps):
        self._total = total_steps
        self._completed = 0
    
    @property
    def completed(self):
        return self._completed
    
    @completed.setter
    def completed(self, value):
        if value < 0:
            self._completed = 0
        elif value > self._total:
            self._completed = self._total
        else:
            self._completed = value
    
    @property
    def percentage(self):
        return (self._completed / self._total) * 100

tracker = ProgressTracker(10)
tracker.completed = 7
print(tracker.percentage)
tracker.completed = 15
print(tracker.completed)
print(tracker.percentage)
    \end{lstlisting}
        \textbf{Ausgabe:}
        \vspace{2cm}

        \textbf{Was ist der Zweck der Setter-Validierung in diesem Code?}
        \vspace{2cm}

        \newpage
  \item \textbf{(4 Pkt.)} Der folgende Code enthält einen häufigen OOP-Fehler. Identifizieren Sie den Fehler, erklären Sie, warum er auftritt, und schreiben Sie den korrigierten Code.
        \begin{lstlisting}
class ChatRoom:
    messages = []
    
    def __init__(self, room_name):
        self.room_name = room_name
    
    def post_message(self, user, text):
        self.messages.append(f"{user}: {text}")

room1 = ChatRoom("General")
room2 = ChatRoom("Sports")
room1.post_message("Alice", "Hello!")
room2.post_message("Bob", "Goal!")
print(f"General: {room1.messages}")
print(f"Sports: {room2.messages}")
    \end{lstlisting}
        \textbf{Was ist der Fehler und warum tritt er auf?}
        \vspace{2.5cm}

        \textbf{Schreiben Sie die korrigierte \texttt{\_\_init\_\_}-Methode:}
        \vspace{4cm}

        \newpage

  \item \textbf{(4 Pkt.)} Vervollständigen Sie die \texttt{Playlist}-Klasse mit korrekter Kapselung. Die Songs-Liste soll privat sein (verwenden Sie \texttt{\_songs}) und nur über eine Property zugänglich sein, die eine Kopie zurückgibt (um externe Modifikation zu verhindern).
        \begin{lstlisting}
class Playlist:
    def __init__(self, name):
        self.name = name
        # TODO: Private _songs als leere Liste initialisieren
 
 

    @property
    def songs(self):
        # TODO: Eine KOPIE der Songs-Liste zurückgeben
 
 

    @property
    def count(self):
        # TODO: Anzahl der Songs zurückgeben
 
 

    def add_song(self, title):
        # TODO: Song nur hinzufügen, wenn nicht bereits vorhanden
        # True zurückgeben wenn hinzugefügt, False bei Duplikat
 
 
 
 
 
 

    def remove_song(self, title):
        # TODO: Song entfernen falls vorhanden
        # True zurückgeben wenn entfernt, False wenn nicht gefunden
 
 
 
 
 
 
    \end{lstlisting}

        \newpage
  \item \textbf{(3 Pkt.)} Betrachten Sie die folgende Vererbungshierarchie:
        \begin{lstlisting}
class Notification:
    def __init__(self, message):
        self.message = message
    
    def send(self):
        return f"Sending: {self.message}"

class EmailNotification(Notification):
    def __init__(self, message, recipient):
        super().__init__(message)
        self.recipient = recipient
    
    def send(self):
        base = super().send()
        return f"{base} to {self.recipient}"

class UrgentEmail(EmailNotification):
    def send(self):
        return "[URGENT] " + super().send()

notif = UrgentEmail("Server down!", "admin@example.com")
print(notif.send())
print(isinstance(notif, Notification))
    \end{lstlisting}

        \begin{enumerate}[label=\alph*)]
          \item Was wird ausgegeben?
                \vspace{1.5cm}

          \item Verfolgen Sie die \texttt{super().send()}-Aufrufe: Welche Methoden werden in welcher Reihenfolge aufgerufen, wenn \texttt{notif.send()} ausgeführt wird?
                \vspace{2cm}

          \item Warum braucht \texttt{UrgentEmail} keine eigene \texttt{\_\_init\_\_}-Methode?
                \vspace{2cm}
        \end{enumerate}
\end{enumerate}

\vfill

\begin{center}
  \rule{8cm}{0.4pt}\\[0.3cm]
  \textbf{Ende der Prüfung}\\[0.2cm]
  \textit{Viel Erfolg!}
\end{center}

\end{document}
