\documentclass[a4paper,12pt]{article}
\usepackage[utf8]{inputenc}
\usepackage[T1]{fontenc}
\usepackage[ngerman]{babel}
\usepackage{amsmath,amssymb}
\usepackage{enumitem}
\usepackage{listings}
\usepackage{xcolor}
\usepackage{geometry}
\usepackage{fancyhdr}
\usepackage[hidelinks]{hyperref}
\usepackage[de-DE]{datetime2}  % Use German date format

\geometry{left=2.5cm, right=2.5cm, top=3.5cm, bottom=3.0cm}

% Define colors and style for code listings
\definecolor{codegreen}{rgb}{0,0.6,0}
\definecolor{codegray}{rgb}{0.5,0.5,0.5}
\definecolor{codepurple}{rgb}{0.58,0,0.82}
\definecolor{backcolour}{rgb}{0.96,0.96,0.96}

\lstdefinestyle{mystyle}{
    backgroundcolor=\color{backcolour},
    commentstyle=\color{codegreen},
    keywordstyle=\color{blue},
    numberstyle=\tiny\color{codegray},
    stringstyle=\color{codepurple},
    basicstyle=\ttfamily\footnotesize,
    breakatwhitespace=false,
    breaklines=true,
    captionpos=b,
    numbers=left,
    numbersep=5pt,
    showspaces=false,
    showstringspaces=false,
    showtabs=false,
    tabsize=2
}
\lstset{style=mystyle}

% Fancy header and footer settings
\pagestyle{fancy}
\fancyhead[L]{Python Programmierung III}
\fancyhead[R]{\today}
\fancyfoot[C]{}
\fancyfoot[R]{\thepage}
\setlength{\headheight}{14.5pt}

\title{Abschlussklausur\\Python Programmierung III}
% \author{[Ihr Name oder Kursname]}
\DTMsetdatestyle{de-DE}
\date{\DTMdisplaydate{2025}{2}{14}{-1}}

\begin{document}

\maketitle

\section*{Allgemeine Hinweise}
Beantworten Sie alle Fragen schriftlich. Es sind keinerlei Hilfsmittel (Computer, Bücher, Internet) erlaubt. Viel Erfolg!

\vspace{1cm}
\begin{center}
    \begin{tabular}{|c|c|c|}
        \hline
        \textbf{Aufgabe}                                & \textbf{Max. Punkte} & \textbf{Erreichte Punkte} \\ \hline
        1. Python Fundamentals \& Environment Setup    & 4                    &                           \\ \hline
        2. Objektorientierte Programmierung (OOP)         & 6                    &                           \\ \hline
        3. Algorithmen, Effizienz und Rekursion           & 6                    &                           \\ \hline
        4. NumPy                                        & 6                    &                           \\ \hline
        5. Pandas                                       & 7                    &                           \\ \hline
        \textbf{Gesamt}                                  & 29                   &                           \\ \hline
    \end{tabular}
\end{center}

\newpage

%%%%%%%%%%%%%%%%%%%%%%%%%%%%%%%%%%%%%%%%%%%%%%%%%%%%%%%%%%%%%%%%%%%%%%
\section{Python Fundamentals \& Environment Setup (4 Punkte)}\label{sec:python}

\begin{enumerate}[label=\arabic*.]
  \item Erklären Sie, was ein virtuelles Python-Environment ist und warum es in Projekten sinnvoll ist. (2 Punkte)

  \item Welche Befehle verwenden Sie, um ein virtuelles Environment mit \texttt{conda} zu erstellen und zu aktivieren? Schreiben Sie den entsprechenden Befehl. (2 Punkte)\\[0.3cm]
  \textbf{Hinweis:} Verwenden Sie z. B. den Befehl zur Erstellung eines neuen Environments namens \texttt{myenv}.
\end{enumerate}

\newpage

%%%%%%%%%%%%%%%%%%%%%%%%%%%%%%%%%%%%%%%%%%%%%%%%%%%%%%%%%%%%%%%%%%%%%%
\section{Objektorientierte Programmierung (OOP) (6 Punkte)}\label{sec:oop}

\begin{enumerate}[label=\arabic*.]
  \item Erklären Sie den Unterschied zwischen statischen Methoden und Klassenmethoden in Python. Geben Sie ein Beispiel, wann eine statische Methode sinnvoll ist. (2 Punkte)

  \item Der folgende Code enthält einen Fehler. Finden Sie den Fehler und korrigieren Sie den Code, sodass die Methode \texttt{beschreiben} den Namen der Person korrekt ausgibt. (2 Punkte)
  \vspace{0.3cm}
  \begin{lstlisting}[language=Python, caption={Fehlersuche in einer Klassenmethode}]
class Person:
    def __init__(self, name):
        self.name = name

    def beschreiben(self):
        print("Die Person heißt " + Name)

p = Person("Laura")
p.beschreiben()
  \end{lstlisting}

  \item Nennen Sie ein praktisches Beispiel, in welchem das Singleton-Muster sinnvoll eingesetzt wird. (2 Punkte)
\end{enumerate}

\newpage

%%%%%%%%%%%%%%%%%%%%%%%%%%%%%%%%%%%%%%%%%%%%%%%%%%%%%%%%%%%%%%%%%%%%%%
\section{Algorithmen, Effizienz und Rekursion (6 Punkte)}\label{sec:algorithms}

\begin{enumerate}[label=\arabic*.]
  \item Vergleichen Sie die Zeitkomplexitäten von Bubble Sort und Quick Sort in Ihren eigenen Worten. (2 Punkte)

  \item Der folgende rekursive Code berechnet die n-te Fibonacci-Zahl. Es fehlt eine Basisfallbedingung. Vervollständigen Sie den Code. (2 Punkte)
  \vspace{0.3cm}
  \begin{lstlisting}[language=Python, caption={Rekursiver Algorithmus zur Berechnung der Fibonacci-Zahl}]
def fibonacci(n):
    if ___________________________:
        return n
    else:
        return fibonacci(n-1) + fibonacci(n-2)

print(fibonacci(6))  % Erwartete Ausgabe: 8
  \end{lstlisting}
  
  \item Beschreiben Sie, warum es bei rekursiven Algorithmen wichtig ist, einen Basisfall zu definieren, und wie Memoization die Effizienz verbessern kann. (2 Punkte)
\end{enumerate}

\newpage

%%%%%%%%%%%%%%%%%%%%%%%%%%%%%%%%%%%%%%%%%%%%%%%%%%%%%%%%%%%%%%%%%%%%%%
\section{NumPy (6 Punkte)}\label{sec:numpy}

\begin{enumerate}[label=\arabic*.]
  \item Erklären Sie, warum NumPy-Arrays für numerische Berechnungen gegenüber Python-Listen bevorzugt werden. (2 Punkte)

  \item Betrachten Sie folgenden Code. Geben Sie die erwartete Ausgabe an und begründen Sie, warum. (1 Punkt)
  \vspace{0.3cm}
  \begin{lstlisting}[language=Python, caption={Broadcasting-Beispiel}]
import numpy as np
a = np.array([1, 2, 3])
b = np.array([[10], [20], [30]])
result = a + b
print(result)
  \end{lstlisting}

  \item Vervollständigen Sie den folgenden Code, um ein 3x3 Array mit Zufallszahlen zwischen 0 und 1 zu erzeugen. (1 Punkt)
  \vspace{0.3cm}
  \begin{lstlisting}[language=Python, caption={Ergänzen Sie den Code}]
import numpy as np
array_random = np.random.______________((3,3))
print(array_random)
  \end{lstlisting}

  \item Was bedeutet der Begriff Broadcasting in NumPy? Geben Sie eine kurze Erklärung und ein Beispiel. (2 Punkte)
\end{enumerate}

\newpage

%%%%%%%%%%%%%%%%%%%%%%%%%%%%%%%%%%%%%%%%%%%%%%%%%%%%%%%%%%%%%%%%%%%%%%
\section{Pandas (7 Punkte)}\label{sec:pandas}

\begin{enumerate}[label=\arabic*.]
  \item Erklären Sie den Unterschied zwischen einer Series und einem DataFrame in Pandas. Wann würden Sie welche Datenstruktur verwenden? (2 Punkte)

  \item Betrachten Sie den folgenden Codeausschnitt. Was ist die Ausgabe, wenn dieser Code ausgeführt wird? (1 Punkt)
  \vspace{0.3cm}
  \begin{lstlisting}[language=Python, caption={DataFrame-Filterung}]
import pandas as pd
data = {'Name': ['Anna', 'Bernd', 'Clara', 'David'],
        'Alter': [28, 34, 29, 42],
        'Stadt': ['Berlin', 'München', 'Hamburg', 'Köln']}
df = pd.DataFrame(data)
print(df[df['Alter'] > 30])
  \end{lstlisting}

  \item Vervollständigen Sie den folgenden Code, der eine CSV-Datei einliest und alle Zeilen ausgibt, in denen der \texttt{total\_bill} größer als 20 ist. (1 Punkt)
  \vspace{0.3cm}
  \begin{lstlisting}[language=Python, caption={Bedingte Filterung im tips-Datensatz}]
df = pd.read_csv("tips.csv")
df_filtered = df[ ___________________________ ]
print(df_filtered.head())
  \end{lstlisting}

  \item Beschreiben Sie, wie Sie in Pandas fehlende Werte erkennen und behandeln können. Erklären Sie den Unterschied zwischen \texttt{fillna()} und \texttt{dropna()}. (2 Punkte)

  \item Vervollständigen Sie den folgenden Code, um eine neue Spalte \texttt{Tip\_Percentage} zu erstellen, die den Prozentsatz des Trinkgelds am Gesamtbetrag berechnet. (1 Punkt)
  \vspace{0.3cm}
  \begin{lstlisting}[language=Python, caption={Ergänzen Sie den Code zur neuen Spalte in tips.csv}]
df = pd.read_csv("tips.csv")
df["Tip_Percentage"] = ______________________________
print(df[["total_bill", "tip", "Tip_Percentage"]].head())
  \end{lstlisting}

  \item Beschreiben Sie in eigenen Worten, wie Sie eine Zeitreihe in Pandas analysieren würden. Welche Schritte sind dabei notwendig? (1 Punkt)
\end{enumerate}

\newpage

%%%%%%%%%%%%%%%%%%%%%%%%%%%%%%%%%%%%%%%%%%%%%%%%%%%%%%%%%%%%%%%%%%%%%%
\section*{Hinweis}
Überprüfen Sie alle Antworten sorgfältig. Ihre Erklärungen und Code-Snippets sollten klar und vollständig formuliert sein.

\vfill
\hrule
\small{Viel Erfolg bei der Klausur!}

\end{document}
