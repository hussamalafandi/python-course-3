\documentclass[a4paper,12pt]{article}
\usepackage[utf8]{inputenc}
\usepackage[T1]{fontenc}
\usepackage[ngerman]{babel}
\usepackage{amsmath,amssymb}
\usepackage{enumitem}
\usepackage{listings}
\usepackage{xcolor}
\usepackage{geometry}
\usepackage{fancyhdr}
\usepackage[hidelinks]{hyperref}
\usepackage[de-DE]{datetime2}

\geometry{left=2.5cm, right=2.5cm, top=3.5cm, bottom=3.0cm}

% Define colors and style for code listings
\definecolor{codegreen}{rgb}{0,0.6,0}
\definecolor{codegray}{rgb}{0.5,0.5,0.5}
\definecolor{codepurple}{rgb}{0.58,0,0.82}
\definecolor{backcolour}{rgb}{0.96,0.96,0.96}

\lstdefinestyle{mystyle}{
    backgroundcolor=\color{backcolour},
    commentstyle=\color{codegreen},
    keywordstyle=\color{blue},
    numberstyle=\tiny\color{codegray},
    stringstyle=\color{codepurple},
    basicstyle=\ttfamily\footnotesize,
    breakatwhitespace=false,
    breaklines=true,
    captionpos=b,
    numbers=left,
    numbersep=5pt,
    showspaces=false,
    showstringspaces=false,
    showtabs=false,
    tabsize=2
}
\lstset{style=mystyle}

% Fancy header and footer settings
\pagestyle{fancy}
\fancyhead[L]{Python Programmierung III}
\fancyhead[R]{\today}
\fancyfoot[C]{}
\fancyfoot[R]{\thepage}
\setlength{\headheight}{14.5pt}

\title{Abschlussklausur\\Python Programmierung III (Alternative)}
% \author{[Ihr Name oder Kursname]}
\DTMsetdatestyle{de-DE}
\date{\DTMdisplaydate{2025}{2}{27}{-1}}

\begin{document}

\maketitle


% --- Student Information Fields ---
\thispagestyle{empty}
\vspace{1cm}
\begin{center}
  \textbf{Name:} \rule{8cm}{0.4pt} \\[0.5cm]
  \textbf{Vorname:} \rule{8cm}{0.4pt} \\[0.5cm]
  \textbf{Unterschrift:} \rule{8cm}{0.4pt}
\end{center}

\vfil

\begin{center}
  \begin{tabular}{|c|c|c|}
    \hline
    \textbf{Aufgabe}     & \textbf{Max. Punkte} & \textbf{Erreichte Punkte} \\ \hline
    \ref{sec:python}     & 4                    &                           \\ \hline
    \ref{sec:oop}        & 6                    &                           \\ \hline
    \ref{sec:algorithms} & 6                    &                           \\ \hline
    \ref{sec:numpy}      & 8                    &                           \\ \hline
    \ref{sec:pandas}     & 7                    &                           \\ \hline
    \textbf{Gesamt}      & 31                   &                           \\ \hline
  \end{tabular}
\end{center}

\newpage

%%%%%%%%%%%%%%%%%%%%%%%%%%%%%%%%%%%%%%%%%%%%%%%%%%%%%%%%%%%%%%%%%%%%%%
\section{Python Fundamentals \& Environment Setup (4 Punkte)}\label{sec:python}

\begin{enumerate}
  \item Erklären Sie, \textbf{was} ein virtuelles Python-Environment ist und \textbf{warum} es in Projekten sinnvoll ist. (2 Punkte)
        \vfil
  \item Welche Befehle verwenden Sie, um ein virtuelles Environment mit \texttt{conda} zu \textbf{erstellen} und zu \textbf{aktivieren}? Schreiben Sie den entsprechenden Befehl. (2 Punkte)\\[0.3cm]
        \textbf{Hinweis:} Verwenden Sie z. B. den Befehl zur Erstellung eines neuen Environments namens \texttt{python\_kurs\_3}.
\end{enumerate}
\newpage

%%%%%%%%%%%%%%%%%%%%%%%%%%%%%%%%%%%%%%%%%%%%%%%%%%%%%%%%%%%%%%%%%%%%%%
\section{Objektorientierte Programmierung (OOP) (6 Punkte)}\label{sec:oop}

\begin{enumerate}
  \item Erklären Sie das Konzept der \textbf{Kapselung} in der objektorientierten Programmierung. Warum ist sie nützlich? Nennen Sie ein Beispiel, wie Kapselung in Python umgesetzt wird. (2 Punkte)

        \vfil

  \item Nennen Sie zwei Entwurfsmuster und erklären Sie einen von ihnen. (2 Punkte)

        \newpage\item Der folgende Code enthält einen Fehler in Bezug auf \textbf{private Attribute} und \textbf{abstrakte Methoden}. Finden Sie die zwei Fehler und erklären Sie, warum sie problematisch sind. (2 Punkte)


        \begin{lstlisting}[language=Python]
from abc import ABC, abstractmethod

class Fahrzeug(ABC):
def __init__(self, marke, modell):
  self.__marke = marke
  self.__modell = modell

@abstractmethod
def beschreibung(self):
  pass

class Auto(Fahrzeug):
def __init__(self, marke, modell, ps):
  super().__init__(marke, modell)
  self.ps = ps

def auto_beschreibung(self):
  return f"{self.__marke} {self.__modell} mit {self.ps} PS"

# Objekt erstellen
auto1 = Auto("BMW", "X5", 300)
print(auto1.beschreibung())
\end{lstlisting}
        \textbf{Hinweis:} Sie müssen keinen Code schrieben, sondern nur die Fehler finden und erklären.
\end{enumerate}

\newpage

%%%%%%%%%%%%%%%%%%%%%%%%%%%%%%%%%%%%%%%%%%%%%%%%%%%%%%%%%%%%%%%%%%%%%%
\section{Algorithmen, Effizienz und Rekursion (6 Punkte)}\label{sec:algorithms}

\begin{enumerate}
  \item Erklären Sie, warum ein iterativer Ansatz manchmal gegenüber einem rekursiven bevorzugt wird. Nennen Sie jeweils einen Vorteil. (2 Punkte)
        \vfil
  \item Implementieren Sie eine rekursive Funktion zur Berechnung von \( a^b \), wobei \( a \) die Basis und \( b \) der Exponent ist. Ergänzen Sie die Basisfallbedingung. (2 Punkte)
        \vspace{0.3cm}
        \begin{lstlisting}[language=Python]
def power(a, b):
    if ___________________________:
        return ___________
    return a * power(a, b - 1)

print(power(2, 3))  # Erwartete Ausgabe: 8 (2^3 = 2 * 2 * 2)
        \end{lstlisting}

        \newpage

  \item Der folgende Python-Code sucht das größte Element in einer Liste. Analysieren Sie den Code und geben Sie die \textbf{Zeitkomplexität} des Algorithmus an. (2 Punkte)

        \vspace{0.3cm}
        \begin{lstlisting}[language=Python]
def find_max(arr):
    max_value = arr[0]
    for i in range(1, len(arr)):
        if arr[i] > max_value:
            max_value = arr[i]
    return max_value

# Beispiel:
print(find_max([3, 7, 2, 9, 5]))  # Erwartete Ausgabe: 9
        \end{lstlisting}
        \textbf{Hinweis:} Bitte geben Sie nur die Zeitkomplexität in Big-O-Notation an.
\end{enumerate}

\newpage

%%%%%%%%%%%%%%%%%%%%%%%%%%%%%%%%%%%%%%%%%%%%%%%%%%%%%%%%%%%%%%%%%%%%%%
\section{NumPy (8 Punkte)}\label{sec:numpy}

\begin{enumerate}
  \item Nennen Sie zwei Funktionen zur Erstellung von NumPy-Arrays und erläutern Sie kurz, worin sie sich unterscheiden. (2 Punkte)

  \item Was bewirkt der Befehl \texttt{np.random.randint(0, 100, (3,4))}? (1 Punkt)

  \item Vervollständigen Sie den folgenden Code, um ein Array mit 10 gleichmäßig verteilten Werten zwischen 5 und 15 zu erzeugen. (1 Punkt)
        \vspace{0.3cm}
        \begin{lstlisting}[language=Python, caption={Ergänzen Sie den Code}]
import numpy as np
array_linspace = np.linspace(____, ____, ____)
print(array_linspace)
  \end{lstlisting}

  \item Was gibt folgender Code aus? Begründen Sie kurz Ihre Antwort. (2 Punkte)
        \vspace{0.3cm}
        \begin{lstlisting}[language=Python, caption={Ausgabevorhersage}]
import numpy as np
a = np.array([2, 4, 6])
print(a * 3)
  \end{lstlisting}

  \item Erklären Sie, was Broadcasting in NumPy bedeutet, und nennen Sie ein typisches Anwendungsbeispiel. (2 Punkte)
\end{enumerate}

\newpage

%%%%%%%%%%%%%%%%%%%%%%%%%%%%%%%%%%%%%%%%%%%%%%%%%%%%%%%%%%%%%%%%%%%%%%
\section{Pandas (7 Punkte)}\label{sec:pandas}

\begin{enumerate}
  \item Erklären Sie den Unterschied zwischen einer \textbf{Series} und einem \textbf{DataFrame} in Pandas. Wann würden Sie welche Datenstruktur verwenden? (2 Punkte)

  \item Welche Methode verwenden Sie, um die ersten 5 Zeilen eines DataFrames anzuzeigen? (1 Punkt)

  \item Vervollständigen Sie den folgenden Code, um den Median der Spalte \texttt{total\_bill} im \texttt{tips}-Datensatz zu berechnen. (1 Punkt)
        \vspace{0.3cm}
        \begin{lstlisting}[language=Python, caption={Berechnung des Medians}]
import pandas as pd
df = pd.read_csv("tips.csv")
median_total = df["total_bill"].____________()
print(median_total)
  \end{lstlisting}

  \item Schreiben Sie ein Code-Snippet, das den \texttt{tips}-Datensatz einliest und alle Zeilen anzeigt, bei denen das Trinkgeld mehr als 10\% des Gesamtbetrags beträgt. (2 Punkte)
        \vspace{0.3cm}
        \begin{lstlisting}[language=Python, caption={Filterung im tips-Datensatz}]
df = pd.read_csv("tips.csv")
df_filtered = df[ ___________________________ ]
print(df_filtered.head())
  \end{lstlisting}

  \item Beschreiben Sie die wesentlichen Schritte, um in Pandas eine Zeitreihe zu analysieren (Umwandlung einer Datumsspalte in ein Datetime-Objekt, Setzen als Index, Resampling). (1 Punkt)
\end{enumerate}

\newpage

%%%%%%%%%%%%%%%%%%%%%%%%%%%%%%%%%%%%%%%%%%%%%%%%%%%%%%%%%%%%%%%%%%%%%%
\section*{Hinweis}
Überprüfen Sie Ihre Antworten sorgfältig. Ihre Erklärungen und Code-Snippets sollten klar und vollständig formuliert sein.

\vfill
\hrule
\vspace{0.3cm}
\small{Viel Erfolg bei der Klausur!}

\end{document}
