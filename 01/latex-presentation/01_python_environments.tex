\documentclass[aspectratio=169]{beamer}
\usetheme{metropolis}
\usepackage{graphicx}
\usepackage{listings}
\usepackage{xcolor}

% Define custom colors
\definecolor{codegreen}{rgb}{0,0.6,0}
\definecolor{codegray}{rgb}{0.5,0.5,0.5}
\definecolor{codepurple}{rgb}{0.58,0,0.82}
\definecolor{backcolour}{rgb}{0.96,0.96,0.96}

\lstdefinestyle{mystyle}{
    backgroundcolor=\color{backcolour},
    commentstyle=\color{codegreen},
    keywordstyle=\color{blue},
    numberstyle=\tiny\color{codegray},
    stringstyle=\color{codepurple},
    basicstyle=\ttfamily\small,
    breakatwhitespace=false,
    breaklines=true,
    captionpos=b,
    % keepspaces=true,
    numbers=left,
    numbersep=5pt,
    showspaces=false,
    showstringspaces=false,
    showtabs=false,
    tabsize=2
}

\lstset{style=mystyle}


\title{Python 3 - Fortgeschrittene Konzepte und Anwendungen}
\subtitle{Python-Umgebungen und Jupyter Notebooks}
\author{Hussam Alafandi}
\date{\today}

\begin{document}

\maketitle

\section{Einführung}

\begin{frame}{Ziele des Tages}
  \begin{itemize}
    \item Verstehen, warum virtuelle Umgebungen nützlich sind.
    \item Erstellung und Verwaltung von virtuellen Umgebungen.
    \item Installation und Verwaltung von Python-Paketen.
    \item Einführung in Jupyter Notebooks.
    \item Arbeiten mit Jupyter-Zellen und Markdown.
  \end{itemize}
\end{frame}

\begin{frame}{Überblick der Tools}
  \begin{itemize}
    \item \textbf{Python:} Die Programmiersprache, die wir heute nutzen.
    \item \textbf{venv:} Erstellen isolierter Umgebungen für Projekte.
    \item \textbf{pip:} Paketmanager für Python-Bibliotheken.
    \item \textbf{Conda:} Alternative zu venv, speziell für wissenschaftliche Pakete.
    \item \textbf{Jupyter Notebook:} Interaktive Umgebung für Python-Code und Visualisierungen.
  \end{itemize}
\end{frame}

\begin{frame}{Warum virtuelle Umgebungen?}
  \begin{itemize}
    \item Vermeidung von Versionskonflikten zwischen Projekten.
    \item Reproduzierbare Entwicklungsumgebung.
    \item Einfache Verwaltung von Abhängigkeiten.
  \end{itemize}
\end{frame}

\section{Virtuelle Umgebungen erstellen}

\begin{frame}[fragile]{Erstellen einer virtuellen Umgebung mit venv}
  \textbf{Schritte:}
  \begin{enumerate}
    \item Erstellen der Umgebung:
\begin{lstlisting}[language=bash]
  python -m venv my_env
\end{lstlisting}
    \item Aktivieren der Umgebung:
\begin{lstlisting}[language=bash]
  source my_env/bin/activate  # Mac/Linux
  my_env\Scripts\activate  # Windows
\end{lstlisting}
    \item Deaktivieren der Umgebung:
\begin{lstlisting}[language=bash]
  deactivate
\end{lstlisting}
  \end{enumerate}
\end{frame}

\begin{frame}[fragile]{Erstellen einer virtuellen Umgebung mit Conda}
  \textbf{Schritte:}
  \begin{enumerate}
    \item Erstellen der Umgebung:
\begin{lstlisting}[language=bash]
  conda create --name my_env python=3.10
\end{lstlisting}
    \item Aktivieren der Umgebung:
\begin{lstlisting}[language=bash]
  conda activate my_env
\end{lstlisting}
    \item Deaktivieren der Umgebung:
\begin{lstlisting}[language=bash]
  conda deactivate
\end{lstlisting}
  \end{enumerate}
\end{frame}

\section{Verwalten von Abhängigkeiten}

\begin{frame}[fragile]{Installieren von Paketen mit pip}
  \begin{itemize}
    \item Installation eines Pakets mit pip:
\begin{lstlisting}[language=bash]
  pip install notebook
\end{lstlisting}
    \item Anzeigen installierter Pakete:
\begin{lstlisting}[language=bash]
  pip list
\end{lstlisting}
    \item Erstellung einer Liste installierter Pakete (Abhängigkeiten):
\begin{lstlisting}[language=bash]
  pip freeze > requirements.txt
\end{lstlisting}
    \item Installation der gespeicherten Abhängigkeiten:
\begin{lstlisting}[language=bash]
  pip install -r requirements.txt
\end{lstlisting}
  \end{itemize}
\end{frame}

\begin{frame}[fragile]{Installieren von Paketen mit Conda}
  \begin{itemize}
    \item Installation eines Pakets mit Conda:
\begin{lstlisting}[language=bash]
  conda install numpy pandas matplotlib
\end{lstlisting}
    \item Anzeigen installierter Pakete:
\begin{lstlisting}[language=bash]
  conda list
\end{lstlisting}
    \item Erstellung einer Umgebungsdatei für Wiederverwendung:
\begin{lstlisting}[language=bash]
  conda env export > environment.yml
\end{lstlisting}
    \item Installation der gespeicherten Umgebung:
\begin{lstlisting}[language=bash]
  conda env create -f environment.yml
\end{lstlisting}
  \end{itemize}
\end{frame}

\section{Einführung in Jupyter Notebooks}

\begin{frame}{Was ist Jupyter Notebook?}
  \begin{itemize}
    \item Eine interaktive Umgebung für Python-Code.
    \item Unterstützt Code, Markdown, Diagramme und mehr.
    \item Perfekt für Datenanalyse und wissenschaftliches Rechnen.
  \end{itemize}
\end{frame}

\begin{frame}[fragile]{Starten von Jupyter Notebook}
  \begin{itemize}
    \item Sicherstellen, dass Jupyter installiert ist:
\begin{lstlisting}[language=bash]
  pip install notebook
\end{lstlisting}
    \item Starten von Jupyter:
\begin{lstlisting}[language=bash]
  jupyter notebook
\end{lstlisting}
  \end{itemize}
\end{frame}

\section{Jupyter Notebooks in VS Code und Google Colab}

\begin{frame}[fragile]{Jupyter Notebooks in VS Code}
  \begin{itemize}
    \item Visual Studio Code unterstützt Jupyter Notebooks nativ.
    \item Installation der Jupyter-Erweiterung in VS Code
    \item Vorteil: Integrierte Debugging- und Entwicklungsumgebung.
  \end{itemize}
\end{frame}

\begin{frame}[fragile]{Jupyter Notebooks in Google Colab}
  \begin{itemize}
    \item Google Colab ist eine cloudbasierte Jupyter-Notebook-Umgebung.
    \item Kein Setup erforderlich – funktioniert direkt im Browser.
    \item Zugriff auf GPUs und TPUs für schnellere Berechnungen.
    \item Laden eines Notebooks in Colab:
          \begin{itemize}
            \item Direkt eine `.ipynb`-Datei hochladen.
            \item Öffnen aus Google Drive oder GitHub.
          \end{itemize}
    \item Beispiel für das Ausführen eines Codes in Colab:
\begin{lstlisting}[language=Python]
  print("Hello Google Colab!")
\end{lstlisting}
  \end{itemize}
\end{frame}

\begin{frame}[standout]
  \huge Fragen?
\end{frame}

\end{document}
