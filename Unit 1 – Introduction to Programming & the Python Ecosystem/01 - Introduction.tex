\documentclass[aspectratio=169]{beamer}
\usetheme{metropolis}
\usepackage{graphicx}
\usepackage{listings}
\usepackage{xcolor}

% Define custom colors
\definecolor{codegreen}{rgb}{0,0.6,0}
\definecolor{codegray}{rgb}{0.5,0.5,0.5}
\definecolor{codepurple}{rgb}{0.58,0,0.82}
\definecolor{backcolour}{rgb}{0.96,0.96,0.96}

\lstdefinestyle{mystyle}{
    backgroundcolor=\color{backcolour},
    commentstyle=\color{codegreen},
    keywordstyle=\color{blue},
    numberstyle=\tiny\color{codegray},
    stringstyle=\color{codepurple},
    basicstyle=\ttfamily\small,
    breakatwhitespace=false,
    breaklines=true,
    captionpos=b,
    % keepspaces=true,
    numbers=left,
    numbersep=5pt,
    showspaces=false,
    showstringspaces=false,
    showtabs=false,
    tabsize=2
}

\lstset{style=mystyle}

\title{Python 3 - Grundlagen und Ökosystem}
\subtitle{Unit 1: Einführung in Programmierung \& Python-Tooling}
\author{Hussam Alafandi}
\date{\today}

\begin{document}

\maketitle

\section{Unit 1: Überblick}

\begin{frame}{Ziele der Unit}
  \begin{itemize}
    \item Verstehen, was Programmierung ist (Problem $\rightarrow$ Algorithmus $\rightarrow$ Code).
    \item Verstehen, wie Python Code ausführt (Interpreter, Skripte, Notebooks).
    \item Entwicklungsumgebungen kennenlernen: Jupyter und VS Code.
    \item Virtuelle Umgebungen erstellen und nutzen (venv, conda).
    \item Pakete installieren und Abhängigkeiten verwalten (pip, requirements.txt).
    \item Erste Python-Kommandos ausführen und Unterschiede Skript vs. Notebook kennen.
  \end{itemize}
\end{frame}

\begin{frame}{Agenda}
  \begin{enumerate}
    \item Was ist Programmierung?
    \item Wie funktioniert Python?
    \item Skripte vs. Notebooks
    \item Entwicklungsumgebungen (Jupyter, VS Code)
    \item Virtuelle Umgebungen (venv, conda)
    \item Pakete \& Abhängigkeiten (pip, requirements.txt)
    \item Praxis \& Aufgaben
  \end{enumerate}
\end{frame}

\section{Einführung in Programmierung}

\begin{frame}{Was ist Programmierung?}
  \begin{itemize}
    \item Programmierung = Problemlösung durch präzise Anweisungen.
    \item Ein Programm ist eine Abfolge von Schritten, die ein Computer ausführt.
    \item Zentraler Ablauf:
      \begin{itemize}
        \item Problem verstehen
        \item Algorithmus entwerfen
        \item Code implementieren
        \item Testen \& iterieren
      \end{itemize}
  \end{itemize}
\end{frame}

\begin{frame}{Von Problem zu Code (Beispiel)}
  \textbf{Problem:} Berechne die Gesamtsumme eines Einkaufs.\\[4pt]
  \textbf{Algorithmus:}
  \begin{enumerate}
    \item Preise erfassen
    \item Summe berechnen
    \item Ergebnis ausgeben
  \end{enumerate}
  \textbf{Code:} (wir starten heute nur mit sehr einfachen Beispielen)
\end{frame}

\section{Wie Python funktioniert}

\begin{frame}{Wie führt Python Code aus?}
  \begin{itemize}
    \item Python ist eine interpretierte Sprache: Code wird durch den \textbf{Interpreter} ausgeführt.
    \item Quellcode-Dateien enden meist auf \texttt{.py}.
    \item Fehler sind normal: Python gibt meist sehr hilfreiche Fehlermeldungen aus.
    \item Interaktiv vs. Datei-basiert:
      \begin{itemize}
        \item Interaktiv: Python REPL (Shell)
        \item Datei-basiert: Skripte (\texttt{python file.py})
      \end{itemize}
  \end{itemize}
\end{frame}

\begin{frame}{Python-Artefakte}
  \begin{itemize}
    \item \textbf{Skript} (\texttt{.py}): Code wird von oben nach unten ausgeführt.
    \item \textbf{Notebook} (\texttt{.ipynb}): Code wird in Zellen ausgeführt (nicht zwingend linear).
    \item \textbf{Pakete}: Wiederverwendbarer Code, den wir installieren können.
    \item \textbf{Virtuelle Umgebung}: Isolierter Raum für Pakete pro Projekt.
  \end{itemize}
\end{frame}

\section{Skripte vs. Notebooks}

\begin{frame}{Skripte vs. Notebooks}
  \begin{itemize}
    \item \textbf{Skripte:}
      \begin{itemize}
        \item Gut für Softwareentwicklung und wiederholbare Abläufe
        \item Saubere Projektstruktur
        \item Einfach zu versionieren (Git)
      \end{itemize}
    \item \textbf{Notebooks:}
      \begin{itemize}
        \item Gut für Exploration, Lernen, Datenanalyse
        \item Mischung aus Code, Text, Output
        \item Gefahr: ``Hidden State'' durch nicht-lineare Ausführung
      \end{itemize}
  \end{itemize}
\end{frame}

\section{Entwicklungsumgebungen}

\begin{frame}{Jupyter Notebook}
  \begin{itemize}
    \item Interaktive Umgebung für Code + Text (Markdown) + Visualisierungen.
    \item Ideal für:
      \begin{itemize}
        \item Lernen
        \item Datenanalyse
        \item Schnelles Prototyping
      \end{itemize}
    \item Zellen:
      \begin{itemize}
        \item Code-Zellen
        \item Markdown-Zellen
      \end{itemize}
  \end{itemize}
\end{frame}

\begin{frame}{VS Code (Warum es sich lohnt)}
  \begin{itemize}
    \item Professioneller Editor für Skripte und Projekte.
    \item Vorteile:
      \begin{itemize}
        \item Autocomplete \& Linting
        \item Debugging
        \item Integriertes Terminal
        \item Git-Integration
        \item Notebook-Support
      \end{itemize}
  \end{itemize}
\end{frame}

\section{Praxis: Installation \& Checks}

\begin{frame}[fragile]{Python Installation prüfen}
  \begin{itemize}
    \item Version prüfen:
\begin{lstlisting}[language=bash]
python --version
python3 --version
\end{lstlisting}
    \item Wo liegt der Interpreter?
\begin{lstlisting}[language=bash]
which python      # Mac/Linux
where python      # Windows
\end{lstlisting}
  \end{itemize}
\end{frame}

\begin{frame}[fragile]{Erste Python-Kommandos (REPL)}
  \begin{itemize}
    \item Python Shell starten:
\begin{lstlisting}[language=bash]
python
\end{lstlisting}
    \item Erste Befehle:
\begin{lstlisting}[language=Python]
print("Hallo Python!")
2 + 3 * 4
\end{lstlisting}
    \item Python Shell verlassen:
\begin{lstlisting}[language=Python]
exit()
\end{lstlisting}
  \end{itemize}
\end{frame}

\section{Virtuelle Umgebungen}

\begin{frame}{Warum virtuelle Umgebungen?}
  \begin{itemize}
    \item Vermeidung von Versionskonflikten zwischen Projekten.
    \item Reproduzierbarkeit (gleiche Pakete $\rightarrow$ gleiche Ergebnisse).
    \item Saubere Abhängigkeiten pro Projekt.
    \item Best Practice in professionellen Teams.
  \end{itemize}
\end{frame}

\begin{frame}[fragile]{Virtuelle Umgebung mit venv}
  \textbf{Schritte:}
  \begin{enumerate}
    \item Umgebung erstellen:
\begin{lstlisting}[language=bash]
python -m venv .venv
\end{lstlisting}
    \item Aktivieren:
\begin{lstlisting}[language=bash]
source .venv/bin/activate      # Mac/Linux
.venv\Scripts\activate         # Windows
\end{lstlisting}
    \item Deaktivieren:
\begin{lstlisting}[language=bash]
deactivate
\end{lstlisting}
  \end{enumerate}
\end{frame}

\begin{frame}[fragile]{Virtuelle Umgebung mit Conda}
  \textbf{Schritte:}
  \begin{enumerate}
    \item Umgebung erstellen:
\begin{lstlisting}[language=bash]
conda create --name py101 python=3.11
\end{lstlisting}
    \item Aktivieren:
\begin{lstlisting}[language=bash]
conda activate py101
\end{lstlisting}
    \item Deaktivieren:
\begin{lstlisting}[language=bash]
conda deactivate
\end{lstlisting}
  \end{enumerate}
\end{frame}

\begin{frame}{venv vs. Conda (Kurzvergleich)}
  \begin{itemize}
    \item \textbf{venv:}
      \begin{itemize}
        \item In Python eingebaut (kein extra Tool nötig)
        \item Sehr gut für ``pure Python'' Projekte
      \end{itemize}
    \item \textbf{Conda:}
      \begin{itemize}
        \item Stärker im wissenschaftlichen Umfeld (native Libraries)
        \item Separater Paketmanager + Environment Manager
      \end{itemize}
  \end{itemize}
\end{frame}

\section{Pakete \& Abhängigkeiten}

\begin{frame}{Standard Library vs. Third-Party}
  \begin{itemize}
    \item \textbf{Standard Library:} Kommt mit Python (z.\,B. \texttt{math}, \texttt{csv}, \texttt{json}).
    \item \textbf{Third-Party Packages:} Müssen installiert werden (z.\,B. \texttt{numpy}, \texttt{pandas}).
    \item Warum das wichtig ist:
      \begin{itemize}
        \item Projekte brauchen reproduzierbare Abhängigkeiten
        \item Teamarbeit erfordert gleiche Setups
      \end{itemize}
  \end{itemize}
\end{frame}

\begin{frame}[fragile]{Pakete installieren mit pip}
  \begin{itemize}
    \item Paket installieren:
\begin{lstlisting}[language=bash]
pip install numpy pandas
\end{lstlisting}
    \item Installierte Pakete anzeigen:
\begin{lstlisting}[language=bash]
pip list
\end{lstlisting}
    \item Abhängigkeiten exportieren:
\begin{lstlisting}[language=bash]
pip freeze > requirements.txt
\end{lstlisting}
    \item Abhängigkeiten installieren:
\begin{lstlisting}[language=bash]
pip install -r requirements.txt
\end{lstlisting}
  \end{itemize}
\end{frame}

\section{Praxis: Skript vs. Notebook}

\begin{frame}[fragile]{Einfaches Skript: \texttt{hello.py}}
\begin{lstlisting}[language=Python]
print("Hallo aus einem Skript!")
x = 2 + 3 * 4
print("Ergebnis:", x)
\end{lstlisting}
\end{frame}

\begin{frame}{Notebook: gleiche Logik, andere Arbeitsweise}
  \begin{itemize}
    \item Im Notebook:
      \begin{itemize}
        \item Code in Zellen ausführen
        \item Markdown für Dokumentation nutzen
        \item Output direkt unter der Zelle
      \end{itemize}
    \item Wichtig:
      \begin{itemize}
        \item Notebooks können ``State'' behalten
        \item Zellen sollten sinnvoll und reproduzierbar ausgeführt werden
      \end{itemize}
  \end{itemize}
\end{frame}

\section{Aufgaben (Unit 1)}

\begin{frame}{Aufgabe 1: Setup \& Dokumentation}
  \begin{itemize}
    \item Python-Version prüfen und dokumentieren.
    \item Virtuelle Umgebung erstellen und aktivieren.
    \item \texttt{numpy} installieren.
    \item Schritte in \texttt{environment\_setup.md} dokumentieren.
  \end{itemize}
\end{frame}

\begin{frame}{Aufgabe 2: Skript vs. Notebook}
  \begin{itemize}
    \item Erstelle \texttt{hello.py} und \texttt{hello.ipynb} mit:
      \begin{itemize}
        \item Ausgabe einer Begrüßung
        \item Einer einfachen Berechnung
      \end{itemize}
    \item Führe beide aus und vergleiche den Ablauf.
  \end{itemize}
\end{frame}

\begin{frame}{Aufgabe 3: Kurzreflexion}
  \begin{itemize}
    \item Beantworte kurz:
      \begin{enumerate}
        \item Was macht der Python Interpreter?
        \item Wann ist ein Notebook sinnvoller als ein Skript?
        \item Warum sind virtuelle Umgebungen nützlich?
      \end{enumerate}
  \end{itemize}
\end{frame}

\begin{frame}[standout]
  \huge Fragen?
\end{frame}

\end{document}
